\section{Conclusion and future work}\label{Conclusion}
For this project a lot of time was needed only to create a prototype of ComeTogether. It was time consuming to find the proper way to implement the communication between the android client and the backend. And there was a lack of knowledge to handle the errors while creating this project on google app engine. 

It was no time to read conference publication or journal publications. Solutions were mostly found on normal blogs.


In a future work new features like a picture upload or a radius search could be implemented and the backend could be moved to a  Platform as a Service provider.
\newline
Da wir bisher im Bachelor-Studium nur ganz wenig mit Android zu tun hatten, mussten wir uns in die \\App-Programmierung
einarbeiten und an kleinen Beispielprojekten die verschiedenen Android-Mechanismen wie z.B. den AsyncTask, ActionBar und
die SQLite-Datenbank kennenlernen bevor wir richtig anfangen konnten die ComeTogether-App zu entwickeln. Es gab auch 
in der Frontend-Entwicklung immer wieder Probleme, die l�sbar waren aber viel Zeit gekostet haben und wir nicht richtig vorangekommen sind. Wir haben jetzt die wichtigsten Grundbausteine f�r die Entwicklung der ComeTogether-App geschaffen, dazu geh�rt u.a. die Kommunikation mit dem RESTful-Webservice, der Zugriff auf die SQLite-Datenbank als auch die \\Einsatzm�glichkeiten von AsyncTasks. Der n�chste Schritt w�re, diese Grundbausteine zu nutzen und Schritt f�r Schritt die restlichen Activities und Views zu erzeugen, die uns bisher fehlen. 
