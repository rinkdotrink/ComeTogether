\begin{lstlisting}[caption=Testabfrage 1, firstnumber=1]{code:uc1}
Select adbc.Produkt.Name, Count(*)
From adbc.Produkt, adbc.Warenkorb_has_Produkt
where adbc.Produkt.PRODUKT_ID = adbc.Warenkorb_has_Produkt.Produkt_PRODUKT_ID
Group by adbc.Produkt.Name;
\end{lstlisting}

\begin{lstlisting}[caption=Testabfrage 2, firstnumber=1]{code:uc2}
SELECT adbc.kunde.Name, SUM(adbc.produkt.Preis)
FROM adbc.kunde,adbc.Produkt, adbc.Warenkorb, adbc.Warenkorb_has_Produkt
WHERE produkt.PRODUKT_ID = warenkorb_has_produkt.Produkt_PRODUKT_ID 
    AND warenkorb_has_produkt.Warenkorb_WARENKORB_ID = warenkorb.WARENKORB_ID
    AND warenkorb.Kunde_KUNDE_ID = kunde.KUNDE_ID    
    AND (warenkorb_has_produkt.Datum BETWEEN '2011-01-01' AND '2011-03-01')
Group by adbc.kunde.Name;
\end{lstlisting}

\begin{lstlisting}[caption=Testabfrage 3, firstnumber=1]{code:uc3}
SELECT Count(DISTINCT(adbc.kunde.KUNDE_ID))
FROM adbc.Kunde, adbc.Warenkorb, adbc.warenkorb_has_produkt
WHERE warenkorb.Kunde_KUNDE_ID = kunde.KUNDE_ID
AND warenkorb.WARENKORB_ID = warenkorb_has_produkt.Warenkorb_WARENKORB_ID
    AND warenkorb_has_produkt.Datum = '2011-01-01';
\end{lstlisting}

\begin{lstlisting}[caption=Testabfrage 4, firstnumber=1]{code:uc4}
SELECT adbc.kunde.Name, SUM(adbc.produkt.Preis)
FROM adbc.kunde,adbc.Produkt, adbc.Warenkorb, adbc.Warenkorb_has_Produkt
WHERE produkt.PRODUKT_ID = warenkorb_has_produkt.Produkt_PRODUKT_ID 
    AND warenkorb_has_produkt.Warenkorb_WARENKORB_ID = warenkorb.WARENKORB_ID
    AND warenkorb.Kunde_KUNDE_ID = kunde.KUNDE_ID
    AND ((warenkorb_has_produkt.Datum = '2011-01-01')
    OR   (warenkorb_has_produkt.Datum = '2011-01-05')
    OR   (warenkorb_has_produkt.Datum = '2011-01-09')    
    OR   (warenkorb_has_produkt.Datum = '2011-02-02')
    OR   (warenkorb_has_produkt.Datum = '2011-02-06')    
    OR   (warenkorb_has_produkt.Datum = '2011-02-10')
    OR   (warenkorb_has_produkt.Datum = '2011-03-10')
    OR   (warenkorb_has_produkt.Datum = '2011-03-14')    
    OR   (warenkorb_has_produkt.Datum = '2011-03-18'))      
Group by adbc.kunde.Name;
\end{lstlisting}

\begin{lstlisting}[caption=Tabellenerzeugung mit Hash-Partitioning, firstnumber=1]{code:createparthash}
SET @OLD_UNIQUE_CHECKS=@@UNIQUE_CHECKS, UNIQUE_CHECKS=0;
SET @OLD_FOREIGN_KEY_CHECKS=@@FOREIGN_KEY_CHECKS, FOREIGN_KEY_CHECKS=0;
SET @OLD_SQL_MODE=@@SQL_MODE, SQL_MODE='TRADITIONAL';

CREATE SCHEMA IF NOT EXISTS `ADBC` DEFAULT CHARACTER SET utf8 COLLATE utf8_general_ci ;

USE `ADBC`;

CREATE  TABLE IF NOT EXISTS `ADBC`.`Kunde` (
  `KUNDE_ID` INT(11) NULL DEFAULT NULL ,
  `Name` VARCHAR(45) NULL DEFAULT NULL ,
  `Kundennummer` VARCHAR(45) NULL DEFAULT NULL )
ENGINE = InnoDB
DEFAULT CHARACTER SET = utf8
COLLATE = utf8_general_ci;

CREATE  TABLE IF NOT EXISTS `ADBC`.`Warenkorb` (
  `WARENKORB_ID` INT(11) NULL DEFAULT NULL ,
  `Kunde_KUNDE_ID` INT(11) NULL DEFAULT NULL )
ENGINE = InnoDB
DEFAULT CHARACTER SET = utf8
COLLATE = utf8_general_ci;

CREATE  TABLE IF NOT EXISTS `ADBC`.`Produkt` (
  `PRODUKT_ID` INT(11) NULL DEFAULT NULL ,
  `Name` VARCHAR(45) NULL DEFAULT NULL ,
  `Preis` INT(11) NULL DEFAULT NULL )
ENGINE = InnoDB
DEFAULT CHARACTER SET = utf8
COLLATE = utf8_general_ci;

CREATE  TABLE IF NOT EXISTS `ADBC`.`Warenkorb_has_Produkt` (
  `Warenkorb_WARENKORB_ID` INT(11) NULL DEFAULT NULL ,
  `Produkt_PRODUKT_ID` INT(11) NULL DEFAULT NULL ,
  `WARENKORB_HAS_PRODUKT_ID` INT(11) NULL DEFAULT NULL ,
  `Datum` DATE NULL DEFAULT NULL )
ENGINE = InnoDB
DEFAULT CHARACTER SET = utf8
COLLATE = utf8_general_ci

    PARTITION BY HASH( MONTH(Datum) )
    PARTITIONS 12;

;


SET SQL_MODE=@OLD_SQL_MODE;
SET FOREIGN_KEY_CHECKS=@OLD_FOREIGN_KEY_CHECKS;
SET UNIQUE_CHECKS=@OLD_UNIQUE_CHECKS;
\end{lstlisting}

\begin{lstlisting}[caption=Tabellenerzeugung mit List-Partitioning, firstnumber=1]{code:createpartlist}
SET @OLD_UNIQUE_CHECKS=@@UNIQUE_CHECKS, UNIQUE_CHECKS=0;
SET @OLD_FOREIGN_KEY_CHECKS=@@FOREIGN_KEY_CHECKS, FOREIGN_KEY_CHECKS=0;
SET @OLD_SQL_MODE=@@SQL_MODE, SQL_MODE='TRADITIONAL';

CREATE SCHEMA IF NOT EXISTS `ADBC` DEFAULT CHARACTER SET utf8 COLLATE utf8_general_ci ;

USE `ADBC`;

CREATE  TABLE IF NOT EXISTS `ADBC`.`Kunde` (
  `KUNDE_ID` INT(11) NULL DEFAULT NULL ,
  `Name` VARCHAR(45) NULL DEFAULT NULL ,
  `Kundennummer` VARCHAR(45) NULL DEFAULT NULL )
ENGINE = InnoDB
DEFAULT CHARACTER SET = utf8
COLLATE = utf8_general_ci;

CREATE  TABLE IF NOT EXISTS `ADBC`.`Warenkorb` (
  `WARENKORB_ID` INT(11) NULL DEFAULT NULL ,
  `Kunde_KUNDE_ID` INT(11) NULL DEFAULT NULL )
ENGINE = InnoDB
DEFAULT CHARACTER SET = utf8
COLLATE = utf8_general_ci;

CREATE  TABLE IF NOT EXISTS `ADBC`.`Produkt` (
  `PRODUKT_ID` INT(11) NULL DEFAULT NULL ,
  `Name` VARCHAR(45) NULL DEFAULT NULL ,
  `Preis` INT(11) NULL DEFAULT NULL )
ENGINE = InnoDB
DEFAULT CHARACTER SET = utf8
COLLATE = utf8_general_ci;

CREATE  TABLE IF NOT EXISTS `ADBC`.`Warenkorb_has_Produkt` (
  `Warenkorb_WARENKORB_ID` INT(11) NULL DEFAULT NULL ,
  `Produkt_PRODUKT_ID` INT(11) NULL DEFAULT NULL ,
  `WARENKORB_HAS_PRODUKT_ID` INT(11) NULL DEFAULT NULL ,
  `Datum` DATE NULL DEFAULT NULL )
    
ENGINE = InnoDB
DEFAULT CHARACTER SET = utf8
COLLATE = utf8_general_ci

PARTITION BY LIST(MONTH(Datum)) (
    PARTITION quartal1 VALUES IN (1,2,3),
    PARTITION quartal2 VALUES IN (4,5,6),
    PARTITION quartal3 VALUES IN (7,8,9),
    PARTITION quartal4 VALUES IN (10,11,12)
);


SET SQL_MODE=@OLD_SQL_MODE;
SET FOREIGN_KEY_CHECKS=@OLD_FOREIGN_KEY_CHECKS;
SET UNIQUE_CHECKS=@OLD_UNIQUE_CHECKS;

\end{lstlisting}

\begin{lstlisting}[caption=Tabellenerzeugung mit Range-Partitioning, firstnumber=1]{code:createpartrange}
SET @OLD_UNIQUE_CHECKS=@@UNIQUE_CHECKS, UNIQUE_CHECKS=0;
SET @OLD_FOREIGN_KEY_CHECKS=@@FOREIGN_KEY_CHECKS, FOREIGN_KEY_CHECKS=0;
SET @OLD_SQL_MODE=@@SQL_MODE, SQL_MODE='TRADITIONAL';

CREATE SCHEMA IF NOT EXISTS `ADBC` DEFAULT CHARACTER SET utf8 COLLATE utf8_general_ci ;

USE `ADBC`;

CREATE  TABLE IF NOT EXISTS `ADBC`.`Kunde` (
  `KUNDE_ID` INT(11) NULL DEFAULT NULL ,
  `Name` VARCHAR(45) NULL DEFAULT NULL ,
  `Kundennummer` VARCHAR(45) NULL DEFAULT NULL )
ENGINE = InnoDB
DEFAULT CHARACTER SET = utf8
COLLATE = utf8_general_ci;

CREATE  TABLE IF NOT EXISTS `ADBC`.`Warenkorb` (
  `WARENKORB_ID` INT(11) NULL DEFAULT NULL ,
  `Kunde_KUNDE_ID` INT(11) NULL DEFAULT NULL )
ENGINE = InnoDB
DEFAULT CHARACTER SET = utf8
COLLATE = utf8_general_ci;

CREATE  TABLE IF NOT EXISTS `ADBC`.`Produkt` (
  `PRODUKT_ID` INT(11) NULL DEFAULT NULL ,
  `Name` VARCHAR(45) NULL DEFAULT NULL ,
  `Preis` INT(11) NULL DEFAULT NULL )
ENGINE = InnoDB
DEFAULT CHARACTER SET = utf8
COLLATE = utf8_general_ci;

CREATE  TABLE IF NOT EXISTS `ADBC`.`Warenkorb_has_Produkt` (
  `Warenkorb_WARENKORB_ID` INT(11) NULL DEFAULT NULL ,
  `Produkt_PRODUKT_ID` INT(11) NULL DEFAULT NULL ,
  `WARENKORB_HAS_PRODUKT_ID` INT(11) NULL DEFAULT NULL ,
  `Datum` DATE NULL DEFAULT NULL )
ENGINE = InnoDB
DEFAULT CHARACTER SET = utf8
COLLATE = utf8_general_ci

PARTITION BY RANGE COLUMNS(Datum) (
    PARTITION quartal1 VALUES LESS THAN ('2011-04-01'),
    PARTITION quartal2 VALUES LESS THAN ('2011-07-01'),
    PARTITION quartal3 VALUES LESS THAN ('2011-10-01'),
    PARTITION quartal4 VALUES LESS THAN MAXVALUE
);

SET SQL_MODE=@OLD_SQL_MODE;
SET FOREIGN_KEY_CHECKS=@OLD_FOREIGN_KEY_CHECKS;
SET UNIQUE_CHECKS=@OLD_UNIQUE_CHECKS;
\end{lstlisting}

\begin{lstlisting}[caption=Tabellenerzeugung mit Sub-Partitioning, firstnumber=1]{code:createpartsub}
SET @OLD_UNIQUE_CHECKS=@@UNIQUE_CHECKS, UNIQUE_CHECKS=0;
SET @OLD_FOREIGN_KEY_CHECKS=@@FOREIGN_KEY_CHECKS, FOREIGN_KEY_CHECKS=0;
SET @OLD_SQL_MODE=@@SQL_MODE, SQL_MODE='TRADITIONAL';

CREATE SCHEMA IF NOT EXISTS `ADBC` DEFAULT CHARACTER SET utf8 COLLATE utf8_general_ci ;

USE `ADBC`;

CREATE  TABLE IF NOT EXISTS `ADBC`.`Kunde` (
  `KUNDE_ID` INT(11) NULL DEFAULT NULL ,
  `Name` VARCHAR(45) NULL DEFAULT NULL ,
  `Kundennummer` VARCHAR(45) NULL DEFAULT NULL )
ENGINE = InnoDB
DEFAULT CHARACTER SET = utf8
COLLATE = utf8_general_ci;

CREATE  TABLE IF NOT EXISTS `ADBC`.`Warenkorb` (
  `WARENKORB_ID` INT(11) NULL DEFAULT NULL ,
  `Kunde_KUNDE_ID` INT(11) NULL DEFAULT NULL )
ENGINE = InnoDB
DEFAULT CHARACTER SET = utf8
COLLATE = utf8_general_ci;

CREATE  TABLE IF NOT EXISTS `ADBC`.`Produkt` (
  `PRODUKT_ID` INT(11) NULL DEFAULT NULL ,
  `Name` VARCHAR(45) NULL DEFAULT NULL ,
  `Preis` INT(11) NULL DEFAULT NULL )
ENGINE = InnoDB
DEFAULT CHARACTER SET = utf8
COLLATE = utf8_general_ci;

CREATE  TABLE IF NOT EXISTS `ADBC`.`Warenkorb_has_Produkt` (
  `Warenkorb_WARENKORB_ID` INT(11) NULL DEFAULT NULL ,
  `Produkt_PRODUKT_ID` INT(11) NULL DEFAULT NULL ,
  `WARENKORB_HAS_PRODUKT_ID` INT(11) NULL DEFAULT NULL ,
  `Datum` DATE NULL DEFAULT NULL )
ENGINE = InnoDB
DEFAULT CHARACTER SET = utf8
COLLATE = utf8_general_ci

    PARTITION BY RANGE COLUMNS(Datum)
    SUBPARTITION BY HASH( TO_DAYS(Datum) ) (
        PARTITION quartal1 VALUES LESS THAN ('2011-04-01') (
            SUBPARTITION s0,
            SUBPARTITION s1,
            SUBPARTITION s2,
            SUBPARTITION s3
        ),
        PARTITION quartal2 VALUES LESS THAN ('2011-07-01') (
            SUBPARTITION s4,
            SUBPARTITION s5,
            SUBPARTITION s6,
            SUBPARTITION s7
        ),
        PARTITION quartal3 VALUES LESS THAN ('2011-10-01') (
            SUBPARTITION s8,
            SUBPARTITION s9,
            SUBPARTITION s10,
            SUBPARTITION s11
        ),
        PARTITION quartal4 VALUES LESS THAN MAXVALUE (
            SUBPARTITION s12,
            SUBPARTITION s13,
            SUBPARTITION s14,
            SUBPARTITION s15
        )
    );


SET SQL_MODE=@OLD_SQL_MODE;
SET FOREIGN_KEY_CHECKS=@OLD_FOREIGN_KEY_CHECKS;
SET UNIQUE_CHECKS=@OLD_UNIQUE_CHECKS;
\end{lstlisting}

\begin{lstlisting}[caption=Indexerstellung als Hash, firstnumber=1]{code:createindexhash}
CREATE INDEX kunde_idx
    USING HASH
    ON adbc.kunde (KUNDE_ID);
    
CREATE INDEX produkt_idx
    USING HASH
    ON adbc.produkt (PRODUKT_ID);
    
CREATE INDEX warenkorb_idx
    USING HASH
    ON adbc.warenkorb (WARENKORB_ID);
    
CREATE INDEX warenkorb_kunde_idx
    USING HASH
    ON adbc.warenkorb (Kunde_KUNDE_ID);
    
CREATE INDEX warenkorb_has_produkt_idx
    USING HASH
    ON adbc.warenkorb_has_produkt (WARENKORB_HAS_PRODUKT_ID);
    
CREATE INDEX warenkorb_has_produkt_Warenkorb_WARENKORB_ID_idx
    USING HASH
    ON adbc.warenkorb_has_produkt (Warenkorb_WARENKORB_ID);

CREATE INDEX warenkorb_has_produkt_Produkt_PRODUKT_ID_idx
    USING HASH
    ON adbc.warenkorb_has_produkt (Produkt_PRODUKT_ID);
\end{lstlisting}

\begin{lstlisting}[caption=Indexerstellung als B-Tree, firstnumber=1]{code:createpartbtree}
CREATE INDEX kunde_idx
    USING BTREE
    ON adbc.kunde (KUNDE_ID);
    
CREATE INDEX produkt_idx
    USING BTREE
    ON adbc.produkt (PRODUKT_ID);
    
CREATE INDEX warenkorb_idx
    USING BTREE
    ON adbc.warenkorb (WARENKORB_ID);
    
CREATE INDEX warenkorb_kunde_idx
    USING BTREE
    ON adbc.warenkorb (Kunde_KUNDE_ID);
    
CREATE INDEX warenkorb_has_produkt_idx
    USING BTREE
    ON adbc.warenkorb_has_produkt (WARENKORB_HAS_PRODUKT_ID);
    
CREATE INDEX warenkorb_has_produkt_Warenkorb_WARENKORB_ID_idx
    USING BTREE
    ON adbc.warenkorb_has_produkt (Warenkorb_WARENKORB_ID);

CREATE INDEX warenkorb_has_produkt_Produkt_PRODUKT_ID_idx
    USING BTREE
    ON adbc.warenkorb_has_produkt (Produkt_PRODUKT_ID);
\end{lstlisting}

