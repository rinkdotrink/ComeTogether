\newpage
\chapter{Arbeitsaufteilung}


\begin{table}[h] \begin{flushleft}  \begin{tabular}{|l||c|c|c|c|c|c|}
\hline
\textbf{Arbeit}		&	\textbf{C. Ochmann}	& \textbf{I. K�rner}  \\ \hline \hline
Abstract   	      &                     & 0       \\
Einleitung  &                             		      & ~\ref{Einleitung} \\
Aufgabenstellung&                                  & ~\ref{Aufgabenstellung}  \\
Forschungsgegenstand&                              & ~\ref{RelevanzDesForschungsgegenstandes} \\ 
akt. Wissensstand&                                      & ~\ref{DerAktuelleWissensstand}  \\ 
PostgreSQL&                                         & ~\ref{PostgreSQL} \\
Was ist ein Ausf�hrungsplan? &                      &~\ref{WasIstEinAusfuehrungsplan} \\
Die Abarbeitung von Abfragen in PostgreSQL&  ~\ref{Abarbeitung}       & \\ 
Ausf�hrungsplan ver�ndern& ~\ref{Ausfuehrungsplan}        & \\ 
Kostenparameter&   ~\ref{Kostenparameter}      & \\ 
Indexe&          ~\ref{Indexe}                                    & \\ 
Indextypen&    ~\ref{Indextypen}     & \\ 
Pl�ne mit Bitmap Index Scan&  ~\ref{Plaene}       & \\ 
Plananalyse&   ~\ref{Plananalyse}      & \\ 
Planvergleich&  ~\ref{Planvergleich}       & \\ 
Statistiken&  ~\ref{Statistiken}       & \\ 
Die drei Scan-Algorithmen&                              & ~\ref{DieDreiScanAlgorithmen} \\ 
Die drei Join-Algorithmen&         											& ~\ref{DieDreiJoinAlgorithmen}\\ 
Datenbankentwurf&         															& ~\ref{Datenbankentwurf} \\ 
Der Datengenerator&         																& ~\ref{DerDatengenerator} \\ 
Datenbankabfragen&         															& ~\ref{Datenbankabfragen} \\ 
Was ist ein Hints-System?&        												 & ~\ref{WasIstEinHintsSystem} \\ 
Planerverwirrung&        																		 & ~\ref{Planerverwirrung} \\ 
Reihenfolge von Joins erzwingen&     										    & ~\ref{ReihenfolgeVonJoinsErzwingen} \\ 
Ausf�hrungspl�ne f�r Queries mit und ohne Index&         & ~\ref{AusfuehrungsplaeneFuerQueriesMitUndOhneIndex} \\ 
Was tun bei langsamen Ausf�hrungspl�nen?&       			  & ~\ref{WasTunBeiLangsamenAusfuehrungsplaenen} \\ 
Zusammenfassung&         																& ~\ref{Zusammenfassung} \\ 
Ausblick&        																					 & ~\ref{Ausblick} \\ 
\hline \hline
\end{tabular} \end{flushleft} \caption{Aufteilung von Kapitel 2} \end{table}

