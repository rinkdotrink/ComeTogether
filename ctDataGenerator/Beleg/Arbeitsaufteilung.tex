\newpage
\chapter{Arbeitsaufteilung}

%Tabelle 1:
\begin{table}[h] \begin{flushleft} \begin{tabular}{|l||c|c|c|c|c|c|}
\hline
\textbf{Arbeit}		&	\textbf{C. Ochmann}	& \textbf{I. K�rner}  \\ \hline \hline

Abstract   	      &                     & 0       \\

\hline \hline
\end{tabular} \end{flushleft} \caption{Aufteilung vom Abstract} \end{table}

%Tabelle 2:

\begin{table}[h] \begin{flushleft}  \begin{tabular}{|l||c|c|c|c|c|c|}
\hline
\textbf{Arbeit}		&	\textbf{C. Ochmann}	& \textbf{I. K�rner}  \\ \hline \hline
Einleitung  &                             		      & ~\ref{Einleitung} \\
Aufgabenstellung&                                  &      			            \\
Forschungsgegenstand&                              &  \\ 
akt. Wissensstand&                                      &   \\ 
Eingesetzte Datenbank &  & \\
Projektplanung &  & \\
Anwendungsf�lle&                                   & \\ 
EasyMock   	&                                      & \\ 
Dependency Injection&                              & \\ 
\hline \hline
\end{tabular} \end{flushleft} \caption{Aufteilung von Kapitel 2} \end{table}

\begin{table}[h] \begin{flushleft} \begin{tabular}{|l||c|c|c|c|c|c|}
\hline
\textbf{Arbeit}		&	\textbf{C. Ochmann}	& \textbf{I. K�rner}  \\ \hline \hline
Datengenerator&                                    & \\
\hline \hline
\end{tabular} \end{flushleft} \caption{Aufteilung von Kapitel 3} \end{table}


\begin{table}\begin{flushleft}\begin{tabular}{|l||c|c|c|c|c|c|}
\hline
\textbf{Arbeit}		&	\textbf{C. Ochmann}	& \textbf{I. K�rner}  \\ \hline \hline
Ausblick &  & \\
\hline \hline
\end{tabular} \end{flushleft} \caption{Aufteilung von Kapitel 6} \end{table}