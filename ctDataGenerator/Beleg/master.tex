\documentclass[a4paper,12pt,oneside,bibtotoc,numbers=noenddot]{scrreprt}

%Pakete
\usepackage[latin9]{inputenc}
\usepackage[ngerman]{babel}
\usepackage{listings}
\usepackage{graphicx}
\usepackage{BachelorThesis}

% Allgemeine Informationen
\newcommand\mytitle{Titel der Arbeit}
\newcommand\myauthor{Name des Autors oder der Autoren}
\newcommand\mydepartment{Informatik und Elektrotechnik}
\newcommand\myinstitute{Hochschule Zittau/G\"{o}rlitz}
\newcommand\mytutor{Name und Titel des betreuenden Professors}
\newcommand\mySecondTutor{Name und Titel des betrieblichen Betreuers}

% Abstracts
\newcommand\mysubject{Das deutsche Abstract.}
\newcommand\mysubjectenglish{The english abstract.}

% PDF-Einstellungen
\hypersetup
{
	pdftitle = \mytitle,
	pdfsubject = \mysubject,
	pdfauthor = \myauthor,
	pdfkeywords = {},
	colorlinks = {true},
	pdfborder = 0 0 0
}

\begin{document}
\nocite{*}

%
\pagenumbering{alph}
\begin{titlepage}
\thispagestyle{empty} 
 \begin{center}
 \vspace{2.0cm} 
 {\bfseries \huge Ausf�hrungsplanoptimierung in PostgreSQL\\}
 \vspace{3.0cm} 
 {\bfseries \huge Belegarbeit\\}
 \vspace{3.0cm}
 {\normalsize eingereicht am Fachbereich\\}
 {\bfseries \Large Informatik\\}
 {\normalsize der Hochschule Zittau/G�rlitz (HAW)\\}
 \vspace{1cm}
 {\normalsize als Pr�fungsleistung im Fach\\}
 {\bfseries \Large Fortgeschrittene Datenbank-Konzepte 2\\}
 \vspace{1cm}
 {\normalsize vorgelegt von:\\}
 {\bfseries \Large Christof Ochmann (35989)\\
 Ingo K�rner (40586)\\}
 \vspace{1cm}
 {\normalsize  G�rlitz, 9. Juli 2012\\}
 \vspace{0.5cm}
 Betreuer:	Prof. ten Hagen\\
 \vfill
\end{center}
\end{titlepage}

%
%% Kurzreferat
\thispagestyle{empty}
\section*{Abstract}\label{Abstract}
Diese Arbeit baut auf das Projekt Datenbankkonfigurationen\footnote{https://dl.dropbox.com/u/608146/ADBC1\%20OLAP.pdf}, dass w�hrend der Vorlesungsreihe ADBC1 erstellt wurde, auf.
Im Projekt Datenbankkonfigurationen wird untersucht, wie sich die Ausf�hrungsgeschwindigkeit von Abfragen steigern l�sst.
Im Projekt Ausf�hrungsplanoptimierung in PostgreSQL wird dar�berhinaus untersucht, welche Ausf�hrungspl�ne f�r bestimmte Konfigurationen und bestimmte Queries erzeugt werden. Und wie sich diese Pl�ne auf die Ausf�hrungs\-geschwin\-digkeit von SQL-Queries auswirken.
In beiden Projekten wird nur der Bereich OLAP f�r Kaltstarts von Abfragen betrachtet.
Diese Arbeit behandelt nur Datenbankkonfigurationen, die Einfluss auf den Ausf�hrungsplan haben. F�r alle Konfigurationen die keinen Einfluss haben, wird der Standardwert von PostgreSQL beibehalten.
Ziel der Arbeit ist, Annahmen �ber Ausf�hrungspl�ne zu treffen, diese theoretisch zu begr�nden und dann auch praktisch zu untersuchen. Anhand der praktischen Untersuchungen werden die aufgestellten Hypothesen best�tigt oder wiederlegt.
F�r wiederlegte Hypothesen wird eine Begr�ndung gesucht.

%\mysubject
%\section*{Abstract}
%\mysubjectenglish

\pagenumbering{Roman}
\tableofcontents
\listoffigures
\lstlistoflistings

\begin{listofacronyms}
\acronym{API}{Application Programming Interface}
\acronym{ACID}{Atomicity, Consistency, Isolation, Durability}
\acronym{BLOB}{Binary Large Object}
\acronym{DBMS}{Database management system}
\acronym{ERD}{Entity-Relationship Diagram}
\acronym{IDE}{Integrated Development Environment}
\acronym{JDK}{Java Development Kit}
\acronym{MyISAM}{My Indexed Sequential Access Method}
\acronym{OLAP}{Online Analytical Processing}
\acronym{OLTP}{Online Transaction Processing}
\acronym{PHP}{Hypertext Preprocessor}
\acronym{SQL}{Structured Query Language}
\acronym{WLAN}{Wireless Local Area Network}

\end{listofacronyms}

\begin{flushleft}
\begin{thebibliography}{sotief}
\bibitem{bib1}{Martin, Robert C. (2008): Clean Code: A Handbook of Agile Software Craftsmanship. Prentice Hall International}

\bibitem{bib2}{Freeman, Eric (2007): Entwurfsmuster von Kopf bis Fu�. O'REILLY}

\bibitem{bib3}{\begin{verbatim}http://www.easymock.org/EasyMock3_0_Documentation.html
Abruf: 21.12.2011\end{verbatim}} 

\bibitem{bib4}{\begin{verbatim}http://dev.mysql.com Abruf (11.01.2012)\end{verbatim}} 
\bibitem{bib5}{D��ler, Rolf (2011): MySQL. bhv}
\bibitem{bib6}{Schwartz, Baron. (2008): High Performance MySQL. O'REILLY}
\bibitem{bib7}{\begin{verbatim}http://code.google.com/p/google-guice Abruf: 11.01.2012\end{verbatim}} 

\end{thebibliography}
\end{flushleft}

\newpage
\pagestyle{chapterStyle}
\pagenumbering{arabic}

\chapter{Theorie}
\section{Einleitung}\label{Einleitung}

\section{Aufgabenstellung}\label{Aufgabenstellung}
In diesem Projekt werden Ausf�hrungspl�ne f�r bestimmte Queries untersucht. Dazu werden Queries ausgew�hlt, die im Bereich OLAP und Data Warehouse angesiedelt sind. Als Grundlage wird ein Datenbankentwurf f�r ein Projekt aus der Vorlesungsreihe XML genommen.
Die Tabellen sollen mit Testdaten gef�llt werden. Dazu ist der Datengenerator aus dem Projekt Datenbankkonfigurationen anzupassen.
Die Abfragen werden auf den gef�llten Tabellen angewendet. Die Ausf�hrungspl�ne, die daf�r erzeugt werden, werden nach Performancegesichtspunkten untersucht.

\chapter{Datengenerator}\label{Datengenerator}
\section{Die Abarbeitung von Abfragen in PostgreSQL}
\begin{enumerate} 
\item \textbf{Empfang des SQL-Befehls} \\
Nachdem der SQL-Befehl �ber eine Netzwerkverbindung �bertragen wurde, findet die Kodierungsumwandlung statt, 
und die weiteren Phasen der Abarbeitung sehen den Befehl in der Serverkodierung. Hierbei gibt es nur sehr
geringe Optimierungsm�glichkeiten. Es k�nnen theoretisch CPU-Zyklen gespart werden, wenn die Clientkodierung 
gleich der Serverkodierung ist, ansonsten wird eine Konvertierung durchgef�hrt. Diese Auswirkungen sind jedoch sehr gering.
Der Parameter \textit{client\_encoding} informiert den Server dar�ber, welche Kodierung die ankommenden Befehle haben 
und welche Kodierung das Anfrageergebnis haben soll, welches an den Client gesendet wird. Die Voreinstellung 
gibt an welche Kodierung der Server intern verwendet.

\item \textbf{Parser} \\
In dieser Abarbeitungsphase wird die kodierte Befehlszeichenkette durch einen internen Parse-Baum dargestellt. 
Des Weiteren wird die Befehlszeichenkette auf semantische Bedingungen �berpr�ft und etwas bearbeitet. Die SQL-Befehle werden dann 
aufgeteilt in sogenannte optimierbare Anweisungen(SELECT, INSERT, UPDATE und DELETE) und Hilfsanweisungen.
Die Hilfsanweisungen werden sp�ter direkt ausgef�hrt und sie erzeugen keine Ausgabe. Dagegen kommen die optimierbaren 
Anweisungen in den Rewriter. F�r den Parser gibt es von der Anwenderseite keine M�glichkeit die Geschwindigkeit 
zu optimieren. 


\item \textbf{Query Rewriter} \\
Der Rewriter wendet die Anfrageumschreibregeln(Query Rewrite Rules) an. Dabei werden die Sichten(Views) und andere
benutzerdefinierte Regeln aufgel�st, in die Anfrage eingebaut und im Parse-Baum ersetzt. Da der Rewriter 
vor dem Planer angesiedelt ist, bekommt der Planer es nicht mit, ob die Anfrage aus einer Sicht kam oder nicht. Mit der 
Erstellung einer Sicht hat man somit keinen Optimierungsvorteil.

\item \textbf{Planer / Optimizer} \\
Der Planer bekommt den m�glicherweise umgeschriebenen Parse-Baum und hat die Aufgabe einen Ausf�hrungsplan(execution plan) zu erstellen, der ebenfalls ein Baum ist.
Der Ausf�hrungsplan beschreibt wie auf die Tabellen zugegriffen werden soll, also welche Indexe und Join-Algorithmen verwendet werden sollen und in welcher Reihenfolge. Es soll m�glichst der optimalste und schnellste Ausf�hrungsplan gefunden werden. 

\item \textbf{Executor} \\ 
Der vom Planer auserw�hlte Ausf�hrungsplan wird vom Executor ausgef�hrt. Dabei werden Zugriffsrechte auf Tabellen und andere Objekte sowie Constraints gepr�ft. Die Laufzeit der Ausf�hrung h�ngt nicht nur davon ab ob der Plan gut ist, sondern auch von der gesamten Systemkonfiguration.

\end{enumerate}

%\chapter{Theoretische Grundlagen}
%Die f\"{u}r den Untersuchungsgegenstand relevanten Themen, die \"{u}ber die
%grundlegenden Studieninhalte hinausgehen; oft auch anwendungsspezifische Aspekte - %ca. 6 Seiten

%\chapter{Ist-Analyse}
%Welche Defizite sollen mit der Arbeit behoben werden, welche nicht? %Pr\"{a}zisierung
%der Zielstellung - ca. 6 Seiten

%\chapter{L\"{o}sungskonzept}
%Wie sollen die Defizite behoben werden? Methoden, fachliche Auseinandersetzung
%mit alternativen Ans\"{a}tzen und Auffassungen, Systembeschreibung (Architektur,
%Vorgehensmodell, \ldots) - ca. 12 Seiten

%\chapter{Implementierung}
%Umsetzung des L\"{o}sungskonzepts, Begr\"{u}ndung der verwendeten Technologien - %ca. 8
%Seiten

%\chapter{Ergebnisse}
%Objektive Bewertung der vorliegenden L\"{o}sung, diverse Testverfahren,
%Nutzerbefragungen - ca. 4 Seiten

%\chapter{Fazit und Ausblick}
%Zusammenfassung s\"{a}mtlicher Ergebnisse in Bezug auf die Zielerf\"{u}llung und
%Vorschl\"{a}ge f\"{u}r weiterf\"{u}hrende Arbeiten - ca. 2 Seiten

\bibliographystyle{alphadin}
\begin{appendix}
\newpage
\pagestyle{appendixAStyle}
\chapter{Codebeispiele}
\begin{lstlisting}[caption=alle Tabellen erstellen, firstnumber=1]{code:TabellenErstellen}
CREATE TABLE "user"
(
  userid bigint,
  name text,
  email text,
  gender text,
  birthday date,
  password text,
  image text
)
WITH (
  OIDS=FALSE
);
ALTER TABLE "user"
  OWNER TO postgres;

CREATE TABLE event
(
  eventid bigint NOT NULL,
  creatorid bigint,
  date date,
  eventname text,
  occasion text,
  location text,
  lon double precision,
  lat double precision,
  description text,
  numbermaleconfirmed int,
  numberfemaleconfirmed int
)
WITH (
  OIDS=FALSE
);
ALTER TABLE event
  OWNER TO postgres;

CREATE TABLE message
(
  messageid bigint,
  eventid bigint,
  senderid bigint,
  recipientid bigint,
  textmessage text,
  date date
)
WITH (
  OIDS=FALSE
);
ALTER TABLE message
  OWNER TO postgres;

CREATE TABLE participation
(
  participationid bigint,
  userid bigint,
  eventid bigint
)
WITH (
  OIDS=FALSE
);
ALTER TABLE participation
  OWNER TO postgres;
\end{lstlisting}

\begin{lstlisting}[caption=Datenimport �ber COPY, firstnumber=1]{code:COPY}
COPY public.Event (eventid, creatorid, date, eventname, occasion, location, lon, lat, description, numbermaleconfirmed, numberfemaleconfirmed) From 'C:\Event.txt' DELIMITER ';';
COPY public.Message (messageid, eventid, senderid, recipientid, textmessage, date) From 'C:\Message.txt' DELIMITER ';';
COPY public.Participation (participationid, userid, eventid) From 'C:\Participation.txt' DELIMITER ';';
COPY public.User (userId, name, email, gender, birthday, password, image) From 'C:\User.txt' DELIMITER ';';
\end{lstlisting}


\begin{lstlisting}[caption=Prim�r- und Fremdschl�ssel hinzuf�gen, firstnumber=1]{code:PrimaryForeignKeys}
ALTER TABLE public.event ADD PRIMARY KEY (eventid);
ALTER TABLE public.message ADD PRIMARY KEY (messageid);
ALTER TABLE public.participation ADD PRIMARY KEY (participationid);
ALTER TABLE public.user ADD PRIMARY KEY (userid);

ALTER TABLE event ADD CONSTRAINT event_creatorid FOREIGN KEY (creatorid) REFERENCES public.user (userid) MATCH FULL;
ALTER TABLE message ADD CONSTRAINT message_eventid FOREIGN KEY (eventid) REFERENCES event (eventid) MATCH FULL;
ALTER TABLE message ADD CONSTRAINT message_senderid FOREIGN KEY (senderid) REFERENCES public.user (userid) MATCH FULL;
ALTER TABLE message ADD CONSTRAINT message_recipientid FOREIGN KEY (recipientid) REFERENCES public.user (userid) MATCH FULL;
ALTER TABLE participation ADD CONSTRAINT participation_userid FOREIGN KEY (userid) REFERENCES public.user (userid) MATCH FULL;
ALTER TABLE participation ADD CONSTRAINT participation_eventid FOREIGN KEY (eventid) REFERENCES event (eventid) MATCH FULL;
\end{lstlisting}


\begin{lstlisting}[caption=Indexe auf Spalten legen, firstnumber=1]{code:Indexe}
CREATE INDEX event_creatorid ON public.event(creatorid);
CREATE INDEX message_eventid ON public.message(eventid);
CREATE INDEX message_senderid ON public.message(senderid);
CREATE INDEX message_recipientid ON public.message(recipientid);
CREATE INDEX participation_userid ON public.participation(userid);
CREATE INDEX participation_eventid ON public.participation(eventid);

CREATE INDEX event_date ON public.event(date);
CREATE INDEX event_eventname ON public.event(eventname);
CREATE INDEX event_occasion ON public.event(occasion);
CREATE INDEX event_location ON public.event(location);
CREATE INDEX event_lon ON public.event(lon);
CREATE INDEX event_lat ON public.event(lat);
CREATE INDEX event_numbermaleconfirmed ON public.event(numbermaleconfirmed);
CREATE INDEX event_numberfemaleconfirmed ON public.event(numberfemaleconfirmed);

CREATE INDEX message_textmessage ON public.message(textmessage);
CREATE INDEX message_date ON public.message(date);

CREATE INDEX user_name ON public.user(name);
CREATE INDEX user_email ON public.user(email);
CREATE INDEX user_gender ON public.user(gender);
CREATE INDEX user_birthday ON public.user(birthday);
\end{lstlisting}
\end{appendix}

\newpage
\chapter{Work Packages}


\begin{table}[h] \begin{flushleft}  \begin{tabular}{|l||c|c|c|c|c|c|}
\hline
\textbf{Work Packages}		&	\textbf{C. Ochmann}	& \textbf{I. K�rner}  \\ \hline \hline
abstract   	      &                     & x       \\
introduction&                                      & x  \\ 
requirements engineering& x & x  \\ 
previous work& & x  \\ 
design&                                      & x  \\ 
miscellaneous& & x  \\ 
programming environment& & x  \\ 
backend  &                             		      & x \\
frontend &  x & \\
maven-android-projekt &  x & \\
database&                              & x \\ 
conclusion&                           x       & x  \\
\hline \hline
\end{tabular} \end{flushleft} \caption{Work Packages} \end{table}



\newpage
\chapter{Eigenst�ndigkeitserkl�rung}
Hiermit erkl�re ich, dass ich diese Arbeit selbst�ndig verfasst habe. Mir ist bekannt, dass jede Form des Plagiats mit der Note 5 (Betrugsversuch) bewertet wird.

\begin{tabular}{@{}p{6.0cm}p{6.0cm}}	  		  		 	
	  		 & \\
	  		 & \\
				  			\textbf{Ochmann, Christof} &               Unterschrift:\\				 
				 &\\
				 & \\
				  			\textbf{K�rner, Ingo}   	&                     Unterschrift:\\			
\end{tabular}
\end{document}
