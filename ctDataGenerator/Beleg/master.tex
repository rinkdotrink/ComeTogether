\documentclass[a4paper,12pt,oneside,bibtotoc,numbers=noenddot]{scrreprt}

%Pakete
\usepackage[latin9]{inputenc}
\usepackage[ngerman]{babel}
\usepackage{listings}
\usepackage{graphicx}
\usepackage{BachelorThesis}

% Allgemeine Informationen
\newcommand\mytitle{Titel der Arbeit}
\newcommand\myauthor{Name des Autors oder der Autoren}
\newcommand\mydepartment{Informatik und Elektrotechnik}
\newcommand\myinstitute{Hochschule Zittau/G\"{o}rlitz}
\newcommand\mytutor{Name und Titel des betreuenden Professors}
\newcommand\mySecondTutor{Name und Titel des betrieblichen Betreuers}

% Abstracts
\newcommand\mysubject{Das deutsche Abstract.}
\newcommand\mysubjectenglish{The english abstract.}

% PDF-Einstellungen
\hypersetup
{
	pdftitle = \mytitle,
	pdfsubject = \mysubject,
	pdfauthor = \myauthor,
	pdfkeywords = {},
	colorlinks = {true},
	pdfborder = 0 0 0
}

\begin{document}
\nocite{*}

%
\pagenumbering{alph}
\begin{titlepage}
\thispagestyle{empty} 
 \begin{center}
 \vspace{2.0cm} 
 {\bfseries \huge Ausf�hrungsplanoptimierung in PostgreSQL\\}
 \vspace{3.0cm} 
 {\bfseries \huge Belegarbeit\\}
 \vspace{3.0cm}
 {\normalsize eingereicht am Fachbereich\\}
 {\bfseries \Large Informatik\\}
 {\normalsize der Hochschule Zittau/G�rlitz (HAW)\\}
 \vspace{1cm}
 {\normalsize als Pr�fungsleistung im Fach\\}
 {\bfseries \Large Fortgeschrittene Datenbank-Konzepte 2\\}
 \vspace{1cm}
 {\normalsize vorgelegt von:\\}
 {\bfseries \Large Christof Ochmann (35989)\\
 Ingo K�rner (40586)\\}
 \vspace{1cm}
 {\normalsize  G�rlitz, 9. Juli 2012\\}
 \vspace{0.5cm}
 Betreuer:	Prof. ten Hagen\\
 \vfill
\end{center}
\end{titlepage}

%
%% Kurzreferat
\thispagestyle{empty}
\section*{Abstract}\label{Abstract}
Diese Arbeit baut auf das Projekt Datenbankkonfigurationen\footnote{https://dl.dropbox.com/u/608146/ADBC1\%20OLAP.pdf}, dass w�hrend der Vorlesungsreihe ADBC1 erstellt wurde, auf.
Im Projekt Datenbankkonfigurationen wird untersucht, wie sich die Ausf�hrungsgeschwindigkeit von Abfragen steigern l�sst.
Im Projekt Ausf�hrungsplanoptimierung in PostgreSQL wird dar�berhinaus untersucht, welche Ausf�hrungspl�ne, f�r bestimmte Konfigurationen und bestimmte Queries erzeugt werden. Und wie sich diese Pl�ne auf die Ausf�hrungs\-geschwin\-digkeit von SQL-Queries auswirken.
In beiden Projekten wird nur der Bereich OLAP betrachtet.
Diese Arbeit behandelt nur Datenbankkonfigurationen, die Einfluss auf den Ausf�hrungsplan haben. F�r alle Konfigurationen die keinen Einfluss haben, wird der Standardwert von PostgreSQL beibehalten.
Ziel der Arbeit ist, Annahmen �ber die Ausf�hrungsgeschwindigkeiten verschiedener Ausf�hrungspl�ne zu treffen, diese theoretisch zu begr�nden und dann durch Messergebnisse praktisch zu belegen. Anhand der Messergebnisse werden die aufgestellten Hypothesen best�tigt oder wiederlegt.
F�r wiederlegte Hypothesen wird eine Begr�ndung gesucht.

%\mysubject
%\section*{Abstract}
%\mysubjectenglish

\pagenumbering{Roman}
\tableofcontents
\listoffigures
\lstlistoflistings

\begin{listofacronyms}
\acronym{DBMS}{Datenbankmanagementsystem}
\acronym{ERD}{Entity-Relationship Diagram}
\acronym{OLAP}{Online Analytical Processing}
\acronym{SQL}{Structured Query Language}

\end{listofacronyms}

\begin{flushleft}
\begin{thebibliography}{sotief}
\bibitem{bib1}{Martin, Robert C. (2008): Clean Code: A Handbook of Agile Software Craftsmanship. Prentice Hall International}

\bibitem{bib2}{Freeman, Eric (2007): Entwurfsmuster von Kopf bis Fu�. O'REILLY}

\bibitem{bib4}{\begin{verbatim}http://www.postgresql.org/ (08.06.2012)\end{verbatim}} 

\bibitem{bib5}{\begin{verbatim}http://wiki.postgresql.org/ (08.06.2012)\end{verbatim}} 



\end{thebibliography}
\end{flushleft}

\newpage
\pagestyle{chapterStyle}
\pagenumbering{arabic}

\chapter{Theorie}
\section{Einleitung}\label{Einleitung}
Ziel dieses Projektes ist zu zeigen, welche Ausf�hrungspl�ne, f�r welche Queries bei welchen Indexarten bei welchen Spalten die Ausf�hrungszeit beschleunigen.

Es wird davon ausgegangen, dass die zur Verf�gung stehenden Indexarten sowie die Schreibweise des Queries, den Ausf�hrungsplan und damit die Ausf�hrungs\-geschwindigkeit ver�ndern k�nnen.
\section{Aufgabenstellung}\label{Aufgabenstellung}
In diesem Projekt werden Ausf�hrungspl�ne f�r bestimmte Queries untersucht. Dazu werden Queries ausgew�hlt, die im Bereich OLAP und Data Warehouse angesiedelt sind. Als Grundlage wird ein Datenbankentwurf f�r ein Projekt aus der Vorlesungsreihe XML genommen.
Die Tabellen sollen mit Testdaten gef�llt werden. Dazu ist der Datengenerator aus dem Projekt Datenbankkonfigurationen anzupassen.
Die Abfragen werden auf den gef�llten Tabellen angewendet. Die Ausf�hrungspl�ne, die daf�r erzeugt werden, werden nach Performancegesichtspunkten untersucht.
Nicht untersucht werden soll die Auswirkung von veralteten Statistiken auf die Ausf�hrungsgeschwindigkeit von Ausf�hrungspl�nen. Alle Abfragen basieren stets auf aktuellen Statistiken.
\section{Relevanz des Forschungsgegenstandes}\label{RelevanzDesForschungsgegenstandes}
Der Forschungsgegenstand dieser Arbeit ist, Annahmen �ber erzeugte Ausf�hr\-ungs\-pl�ne von Abfragen zu treffen und gegebenfalls wiederlegte Annahmen zu erkl�ren.
Der Forschungsgegenstand ist relevant, da bisher keine konkreten Aus\-f�hr\-ungs\-pl�ne f�r die gew�hlten Abfragen vorliegen. Ziel dieser Foschung ist es, Ausf�hrungspl�ne zu finden, die die h�chste Ausf�hrungsgeschwindigkeit f�r alle Abfragen bringt. 

Um die optimalen Ausf�hrungspl�ne zu finden, muss sich vertiefend in eine PostgreSQL
eingearbeitet werden. Das geschieht z.B. unter Zuhilfenahme von B�chern und
Online-Ressourcen. In diesen Medien ist der Forschungsstand zur Erstellung von
Ausf�hrungspl�nen dokumentiert. Bei der Anpassung des Datengenerators m�ssen
dar�berhinaus technische Probleme gel�st werden.
\section{Der aktuelle Wissensstand}\label{DerAktuelleWissensstand}
Noch nicht vorhandene Kenntnisse �ber die Erstellung von Ausf�hrunspl�nen werden haupts�chlich aus Onlineressourcen bezogen. Prim�rliteratur zur gew�hlten Datenbank
ist unter www.postgresql.org zu finden. Unter dieser Adresse ist das gew�hlte
DBMS dokumentiert. Auf dieser Seite wird eine Einf�hrung in PostgreSQL gegeben.
Es gibt Installationsanleitungen, eine umfassende Dokumentation von PostgreSQL und Werkzeuge die ein effizienteres Arbeiten mit PostgreSQL erm�glichen.
\section{PostgreSQL}\label{PostgreSQL}
Da im Projekt Datenbankkonfigurationen bereits viel Erfahrung mit MySQL gesammelt wurde, wird im aktuellen Projekt nun nach einem DBMS gesucht, mit qualitativ �hnlichen Eigenschaften zu MySQL. Ziel ist dabei auch andere DBMS kennen zu lernen. F�r die Untersuchung der Ausf�hrungspl�ne wird dazu PostgreSQL gew�hlt, ein freies, objektrelationales Datenbankmanagementsystem f�r das mit PgAdmin auch eine grafische Benutzeroberfl�che zur Verf�gung steht.
\section{Was ist ein Ausf�hrungsplan?}\label{WasIstEinAusfuehrungsplan}
In einem Ausf�hrungsplan wird beschrieben, in welchen Schritten ein DBMS einen Query ausf�hrt. Auch die Reihenfolge der Schritte wird dabei angegeben.
Da in einem Query nur beschrieben steht, was im Ergebnis gew�nscht ist, aber kein Algorithmus, wie dieses Ergebnis errechnet werden soll, gibt es viele verschiedene M�glichkeiten, zu dem Ergebnis zu kommen. Das DBMS soll dabei den Ausf�hrungsplan finden, der das Ergebnis effizient errechnet.

Technisch betrachtet, ist der Ausf�hrungsplan ein Baum. Die Abarbeitung des Ausf�hrungsplans beginnt bei seinen Bl�ttern. In den Bl�ttern des Baumes stehen die Zugriffspfade (access paths).
Der Zugriffspfad wird durch den Scan-Algorithmus angegeben. Ein Scan-Algorithmus gibt an, wie eine Tabelle durchlaufen wird.
\section{Die drei Scan-Algorithmen}\label{DieDreiScanAlgorithmen}
Ein Scan-Algorithmus arbeitet immer nur auf einer einzelnen Tabelle. In Postgre\-SQL gibt es die folgenden drei Scan-Algorithmen:

\begin{enumerate}
\item sequential scan (full table scan) \\
Der Inhalt der Tabelle wird komplett gelesen. Er wird blockweise vom Sekund�rspeicher wie z.B. einer Festplatte in den Arbeitsspeicher geholt.

\item index scan \\
Hat eine Tabelle einen Index, kann er verwendet werden, um die Tupel sortiert zu lesen. Bei einem Index Scan werden Bl�cke auch mehrmals gelesen, wenn der Inhalt der Tabelle nicht auch sortiert in den Bl�cken vorliegt. Das ist relativ teuer und nur f�r kleine Treffermengen geeignet. Ein Index-Scan eignet sich bei einer hohen Selektivit�t eines Selects.

\item bitmap index scan \\
Hier wird der Index gescannt und ein Bitmap mit den getroffenen Blocknummern erzeugt. Das Bitmap der Blocknummern wird dann aufsteigend sortiert. Die Tabelle wird anhand der sortierten Bitmap-Blocknummern aufsteigend gescannt. Das ist nur m�glich, wenn Indexe f�r die betreffenden Spalten existieren.
\end{enumerate}
\section{Die drei Join-Algorithmen}\label{DieDreiJoinAlgorithmen}
In den Knoten des Ausf�hrungsplanes stehen Datenbankoperatoren wie Projek\-tion, Selektion, Kreuzprodukt, Vereinigung, Differenz oder Umbenennung.

Die Hintereinanderausf�hrung der Operationen kartesisches Produkt und Selektion wird Join genannt. Joins werden im DBMS intern �ber Join-Algorithmen realisiert. Join-Algorithmen verkn�pfen Tabellen paarweise miteinander.

In PostgreSQL gibt es drei grundlegende JOIN-Algorithmen:

\begin{enumerate}
\item nested loop join \\
F�r jede Zeile aus der treibenden Tabelle wird die innere Tabelle einmal durchlaufen. Wenn die innere Tabelle indiziert ist, kann sie mit einem Index-Scan durchlaufen werden.
Ein Nested Loop Join kann sehr kostenintensiv werden, wenn er die innere Tabelle mehrmals lesen muss.
Wenn auf der inneren Tabelle ein Index-Scan erfolgen kann, oder die innere Tabelle sehr klein ist, kann sich diese durch vorherige Anfragen im Cache befinden und so schnell abgearbeitet werden.
Wenn es sich um einen sequentiellen Scan handelt, der alle Zeilen vergleicht, dann muss die innere Tabelle u.U. so oft von der Festplatte gelesen werden, wie die treibende Tabelle Zeilen hat.
Wird nur die erste �bereinstimmende Zeile gesucht, ist der Nested Loop Join schneller als andere Joins, die vorher erst ihr komplettes Ergebnis berechnen m�ssen, bevor sie den ersten Treffer zur�ckgeben k�nnen.
Der Nested Loop Join verlangt vor dem Query keinerlei Investition, wie Hashing oder Sortierung.

\item hash join \\
F�r einen Hashjoin m�ssen beide Tabellen als Hash-Tabelle vorliegen. Das setzt vorraus, dass bevor ein Hash Join eingesetzt werden kann, die Tabellen durchgehasht wurden, d.h. ein Hashwert f�r das sp�tere Join-Attribut gebildet wird. Wie der Nested Loop Join ist der Hash-Join besonders performant, wenn der Gr��enunterschied zwischen treibender Tabelle und innerer Tabelle gro� ist und die kleinere Tabelle komplett in den Speicher passt. Dazu wird bei der Ausf�hrung des Hash-Joins die kleinere Hashtabelle in den Arbeitsspeicher geladen, mit dem JOIN-Attribut als Schl�ssel. Dann wird die gr��ere Tabelle gescannt und jeder gefundene Wert wird als Schl�ssel f�r die kleinere Hashtable benutzt.
Ein Hash-Join kann nur dann verwendet werden, wenn die Spalten mit dem = Operator verglichen werden.
Der Performancezugewinn des Hash-Join wird durch einen h�heren Sekund�rspeicherbedarf erkauft, denn die Hashtabellen werden im Tempspace materialisiert. Bei einem Hash Join muss immer ein Materialize erfolgen, der die Buckets erzeugt.
Der Hash-Join eignet sich auch dann, wenn alle L�sungen gebraucht werden und nicht nur z.B. die ersten zehn.

\item merge join \\
Bevor ein Merge Join ausgef�hrt werden kann, m�ssen beide Tabellen nach dem Join-Attributen sortiert werden. Liegen beide Tabellen sortiert vor, werden bei einem Merge Join beide Tabellen parallel gescannt und passende Zeilen werden zusammengef�gt. Sowohl die treibende als auch die innere Tabelle muss nur einmal gescannt werden. Die vorrangegangene Sortierung erfolgt in einem extra Sortierschritt oder durch die Verwendung eines Index, falls das Feld indiziert ist und der JOIN �ber dieses Attribut erfolgt.
Die Sortierung im ersten Fall ist teurer als wenn ein entsprechender Index vorhanden ist.
Werden mehrere Merge-JOINS �ber dasselbe Attribut hintereinander ausgef�hrt, muss nur einmal sortiert werden.
Wie bei dem Hash-Join wird der Performancezugewinn des Merge Joins mit einem h�heren Sekund�rspeicherbedarf erkauft, denn beide Tabellen m�ssen sortiert im Tempspace vorliegen. Der Merge Join ist wie der Hash-Join vor allem dann interessant, wenn alle L�sungen gefunden werden sollen.
Anders als bei Nested Loop Join und Hash-Join spielt bei einem Merge Join der Gr��enunterschied der beiden zu joinenden Tabellen f�r die Ausf�hrungsgeschwindigkeit keine Rolle.
\end{enumerate}

\chapter{Umsetzung}\label{Umsetzung}

\section{Datenbankentwurf}\label{Datenbankentwurf}
In Abbildung \ref{fig:EER-Diagramm} auf Seite \pageref{fig:EER-Diagramm} ist der verwendete Datenbankentwurf zu sehen. Auf ihm werden die zu entwickelnden Queries gefahren.

\begin{figure}[htp]
\centering
\includegraphics[width=1\textwidth]{Ingo/Bilder/EER-Diagramm.png}
\caption{EER-Diagramm}
\label{fig:EER-Diagramm}
\end{figure}
\section{Der Datengenerator}\label{DerDatengenerator}
�ber den grunds�tzlichen Aufbau des Datengenerators wird im Projekt Datanbankkonfigurationen\footnote{https://dl.dropbox.com/u/608146/ADBC1\%20OLAP.pdf} eingegangen.

Um die Daten schneller in die Tabellen einzuf�gen, werden im Datengenerator die generierten Testdaten anders als in DB-Writer nicht mehr �ber Prepared Statements in die Tabellen eingef�gt, sondern �ber die write-Methode von java.io.Writer in eine Datei geschrieben. Um das umzusetzen, wurde die Komponente DB-Writer durch eine Writer-Komponente ersetzt.

Da sich auch das Data Model in diesem Projekt ge�ndert hat, m�ssen weitere Komponenten des Generators angepasst werden.
Die angepassten Komponenten des Generators zeigt Abbildung \ref{fig:KomponentendiagrammDatengenerator} auf Seite \pageref{fig:KomponentendiagrammDatengenerator}.

\begin{figure}[htp]
\centering
\includegraphics[width=1\textwidth]{Ingo/Bilder/Komponentendiagramm.png}
\caption{Komponentendiagramm Datengenerator}
\label{fig:KomponentendiagrammDatengenerator}
\end{figure}

Die Daten werden im CSV-Format in die jeweilige Datei geschrieben, um sie mit dem Copy-Befehl\footnote{http://www.pgadmin.org/docs/1.4/pg/sql-copy.html} von PostgreSQL in die Tabelle laden zu k�nnen.

\begin{lstlisting}[caption=COPY, firstnumber=1]{code:COPY}
COPY public.User (userId, name, email, gender, birthday, password, image) From 'C:\User.txt' DELIMITER ';'
\end{lstlisting}

Dadurch ergibt sich bei einem Umfang von 3,1 Mio generierter Zeilen im Schnitt eine Zeitersparnis um den Faktor zehn - 66 Sekunden f�r die Generierung und den Import der CSV-Datei in die Tabellen mit COPY, zu 687 Sekunden mit prepared statements.

Das Projekt liegt als Maven-Eclipse-Projekt unter:
https://github.com/rinkdotrink/ComeTogether.git
\section{Datenbankabfragen}\label{Datenbankabfragen}

Query 1: Alle user anzeigen, die am event mit dem eventnamen "`event1"' teilnehmen.

\begin{lstlisting}[caption=Query 1, firstnumber=1]{code:Query1}
Select u.userId, u.name, u.email, u.gender, u.birthday
from public.user u, public.event e, public.participation p
where e.eventname = 'event1'
AND e.eventid = p.eventid
AND p.userid = u.userid;
\end{lstlisting}

Query 2: Alle weiblichen user anzeigen, die zwischen 1986 und 1992 geboren worden und an einem event teilnehmen, dass  zwischen dem 01.01.2013 und dem 01.03.2013 stattfindet, und bei dem numberMaleConfirmed / numberFemaleConfirmed kleiner 0,5 ist.

\begin{lstlisting}[caption=Query 2, firstnumber=1]{code:Query2}
Select u.userId, u.name, u.email, u.gender, u.birthday
from public.user u, public.event e, public.participation p
where u.gender = 'w'
AND u.birthday between '01.01.1986' AND '31.12.1992'
AND e.date between '01.01.2013' AND '01.03.2013'
AND e.eventid = p.eventid
AND p.userid = u.userid
AND e.numberMaleConfirmed / e.numberFemaleConfirmed < 0.5;
\end{lstlisting}

Query 3: Die Anzahl der textmessages gruppiert und absteigend sortiert nach numberFemaleConfirmed, die zwischen dem 01.01.2010 und dem 31.12.2012 geschrieben wurden, in denen das Wort Salsa vorkommt, deren Empf�nger m�nnlich sind und zwischen 1972 und 1982 geboren wurden, deren Absender weiblich sind und die zwischen 1986 und 1992 geboren worden und an einem event teilnehmen, dass  zwischen dem 01.01.2013 und dem 01.03.2013 in einem 100km Radius zu der GPS-Koordinate 11.5833 45.1500 stattfindet und bei dem numberMaleConfirmed / numberFemaleConfirmed kleiner 0,5 ist. Das Ergebnis soll nur die ersten f�nf Treffer liefern.

\begin{lstlisting}[caption=Query 3, firstnumber=1]{code:Query3}
Select e.numberFemaleConfirmed, Count(m.messageid) as anzahlMessages
from public.user u, public.event e, public.participation p, public.message m
where u.gender = 'w'
AND u.birthday between '01.01.1986' AND '31.12.1992'
AND e.date between '01.01.2013' AND '01.03.2013'
AND m.date between '01.01.2010' AND '31.12.2012'
AND m.textmessage like '%Salsa'
AND e.eventid = p.eventid
AND p.userid = u.userid
AND m.senderId = u.userId
AND e.numberMaleConfirmed / e.numberFemaleConfirmed < 0.5
AND DEGREES(acos(cos(RADIANS(90-e.lat))*cos(RADIANS(90-11.5833))+sin(RADIANS(90-e.lat))*
sin(RADIANS(90-11.5833))*cos(RADIANS(e.lon-45.1500))))/360*40074 < 100
AND m.recipientId in
(Select u2.userId
from public.user u2
where u2.gender = 'm'
AND u2.birthday between '01.01.1972' AND '31.12.1982'
)
Group by e.numberFemaleConfirmed
Order by e.numberFemaleConfirmed DESC
Limit 5
\end{lstlisting}
\section{Was ist ein Hints-System?}\label{WasIstEinHintsSystem}
In anderen DBMS wie DB2 oder Oracle kann der Optimizer durch Hinweise (hints) dazu gebracht werden, seinen Ausf�hrungsplan zu ver�ndern.

Der Query in Listing \ref{STRAIGHTJOIN} auf Seite \pageref{STRAIGHTJOIN} gibt in MySQL dem Optimizer den Hinweis, die Tabellen so miteinander zu verkn�pfen, wie sie definiert wurden: Es wird tabA als treibende Tabelle und tabB als innere Tabelle verwendet.

\begin{lstlisting}[caption=STRAIGHT JOIN, firstnumber=1, label=STRAIGHTJOIN]{code:STRAIGHTJOIN}
SELECT STRAIGHT_JOIN *
FROM tabA a, tabB b
WHERE a.id = b.id;
\end{lstlisting}

Wenn MySQL den falschen Index aus einer Menge von m�glichen Indexen nimmt, kann z.B. wie in Listing \ref{USEINDEX} dem Optimizer der Hinweis gegeben werden, nur die im folgenden angegebenden Indexe f�r die Abfrage zu verwenden.

\begin{lstlisting}[caption=USE INDEX, firstnumber=1, label=USEINDEX]{code:USEINDEX}
SELECT * FROM table1 USE INDEX (col1_index,col2_index)
  WHERE col1=1 AND col2=2 AND col3=3;
\end{lstlisting}

Mit dem Hinweis IGNORE INDEX $(col2\_index)$ in Listing \ref{IGNOREINDEX} wird der Optimizer veranlasst, den Index $col3\_index$ f�r die Abfrage nicht zu verwenden.

\begin{lstlisting}[caption=IGNORE INDEX, firstnumber=1, label=IGNOREINDEX]{code:IGNOREINDEX}
SELECT * FROM table1 IGNORE INDEX (col3_index)
  WHERE col1=1 AND col2=2 AND col3=3;
\end{lstlisting}

In PostgreSQL gibt es kein hints-System\footnote{http://wiki.postgresql.org/wiki/OptimizerHintsDiscussion}. Es sind in einem Query keine Konstrukte vorgesehen, dem Planner Hinweise zu geben, wie er den Ausf�hrungsplan erstellen soll.

�berholte Hints - wenn z.B. der Hint f�r die aktuelle Tabellengr��en ungeeignet ist - oder die fehlerhafte Anwendung von Hints - k�nnen zu suboptimalen Ausf�hrungspl�nen f�hren und somit die Ausf�hrungsgeschwindigkeit negativ beeinflussen. Die Fehlerquelle "`menschliches Versagen"' beim Schreiben von Hints wird durch den Verzicht auf ein Hints-System reduziert.

Ohne ein Hints-System wird es allerdings auch schwieriger den Planer verschiedene Ausf�hrungspl�ne erzeugen zu lassen. Und der Planer ist mehr auf sich allein gestellt, der er keine Hilfe von menschlicher Seite in Form von Hints erwarten kann.

Auch wenn der Schreiber des Queries sich auf den Planer verlassen muss, sollte er trotzdem ein paar Dinge beachten, auf die im Folgenden eingegangen wird.
\input{Ingo/Planerverwirrung}
\section{Reihenfolge von Joins erzwingen}\label{ReihenfolgeVonJoinsErzwingen}
Um den Planner zu der angegebenen Join-Reihenfolge zu zwingen, kann das

$join\_collapse\_limit$\footnote{http://www.postgresql.org/docs/current/interactive/explicit-joins.html} auf 1 gesetzt werden.

Um den Planner zu zwingen, Subqueries nicht in einen JOIN umzuandeln, kann das $from\_collapse\_limit$ auf 1 gesetzt werden.

Der Standardwert f�r $join\_collapse\_limit$ und $from\_collapse\_limit$ ist acht. Bei z.B. zw�lf zu verkn�pfenden Tabellen wird keine vollst�ndige Suche mehr nach der besten Joinreihenfolge ausgef�hrt, sondern eine wahrscheinlichkeitstheoretische genetische Suche die nur noch eine begrenzte Zahl von m�glichen Joinreihenfolgen betrachtet. Die genetische Suche braucht weniger Zeit als die vollst�ndige Suche, findet aber nicht zwangsl�ufig die bestm�gliche Joinreihenfolge.
Ab welchem Schwellwert die genetische Suche aktiv wird, kann bei $geqo\_threshold$ gesetzt werden. Der Standardwert ist zw�lf.

Auch wenn $join\_collapse\_limit$ auf den Wert eins gesetzt wird, wird bei folgendem Select die JOIN-Reihenfolge vom Planer bestimmt:

\begin{lstlisting}[caption=Reihenfolge vom Planer bestimmt, firstnumber=1, label=PlanerBestimmt]{code:PlanerBestimmt}
SELECT * FROM a, b, c WHERE a.id = b.id AND b.ref = c.id;
\end{lstlisting}

Erst wenn zwei Tabellen ausdr�cklich mit dem Wort JOIN verkn�pft werden, zwingt das den Planner, diese zwei Tabellen in der gegebenen Reihenfolge zu verkn�pfen:

\begin{lstlisting}[caption=Reihenfolge vom Entwickler bestimmt, firstnumber=1, label=EntwicklerBestimmt]{code:EntwicklerBestimmt}
SELECT * FROM a CROSS JOIN b CROSS JOIN c WHERE a.id = b.id AND b.ref = c.id;
SELECT * FROM a JOIN (b JOIN c ON (b.ref = c.id)) ON (a.id = b.id);
\end{lstlisting}
\section{Ausf�hrungspl�ne f�r Queries mit und ohne Index}\label{AusfuehrungsplaeneFuerQueriesMitUndOhneIndex}

Query 1: Alle user anzeigen, die am event mit dem eventnamen "`event1"' teilnehmen.

\begin{lstlisting}[caption=Query 1, firstnumber=1, label=Query1]{code:Query1}
Select u.userId, u.name, u.email, u.gender, u.birthday
from public.user u, public.event e, public.participation p
where e.eventname = 'event1'
AND e.eventid = p.eventid
AND p.userid = u.userid;
\end{lstlisting}

\begin{lstlisting}[caption=Ausf�hrungsplan Query 1, firstnumber=1, label=aQuery1]{code:aQuery1}
"Hash Join  (cost=254.19..619.78 rows=3595 width=108)"
"  Hash Cond: (p.userid = u.userid)"
"  ->  Hash Join  (cost=231.39..467.41 rows=1498 width=8)"
"        Hash Cond: (p.eventid = e.eventid)"
"        ->  Seq Scan on participation p  (cost=0.00..160.64 rows=9664 width=16)"
"        ->  Hash  (cost=231.00..231.00 rows=31 width=8)"
"              ->  Seq Scan on event e  (cost=0.00..231.00 rows=31 width=8)"
"                    Filter: (eventname = 'event1'::text)"
"  ->  Hash  (cost=16.80..16.80 rows=480 width=108)"
"        ->  Seq Scan on "user" u  (cost=0.00..16.80 rows=480 width=108)"
\end{lstlisting}

mit Prim�r- und Fremdschl�sseln f�r alle Tabellen:

\begin{lstlisting}[caption=Prim�r- und Fremdschl�ssel setzen, firstnumber=1, label=PrimaerUndFremdschluessel]{code:PrimaerUndFremdschluessel}
ALTER TABLE public.event ADD PRIMARY KEY (eventid);
ALTER TABLE public.message ADD PRIMARY KEY (messageid);
ALTER TABLE public.participation ADD PRIMARY KEY (participationid);
ALTER TABLE public.user ADD PRIMARY KEY (userid);
ALTER TABLE event ADD CONSTRAINT event_creatorid FOREIGN KEY (creatorid) REFERENCES public.user (userid) MATCH FULL;
ALTER TABLE message ADD CONSTRAINT message_eventid FOREIGN KEY (eventid) REFERENCES event (eventid) MATCH FULL;
ALTER TABLE message ADD CONSTRAINT message_senderid FOREIGN KEY (senderid) REFERENCES public.user (userid) MATCH FULL;
ALTER TABLE message ADD CONSTRAINT message_recipientid FOREIGN KEY (recipientid) REFERENCES public.user (userid) MATCH FULL;
ALTER TABLE participation ADD CONSTRAINT participation_userid FOREIGN KEY (userid) REFERENCES public.user (userid) MATCH FULL;
ALTER TABLE participation ADD CONSTRAINT participation_eventid FOREIGN KEY (eventid) REFERENCES event (eventid) MATCH FULL;
\end{lstlisting}

\begin{lstlisting}[caption=Ausf�hrungsplan Query 1 mit Prim�r- und Fremdschl�sseln, firstnumber=1, label=aQuery1PrimaerUndFremdschluessel]{code:aQuery1PrimaerUndFremdschluessel}
"Nested Loop  (cost=279.63..495.98 rows=50 width=108)"
"  ->  Hash Join  (cost=279.63..481.63 rows=50 width=8)"
"        Hash Cond: (p.eventid = e.eventid)"
"        ->  Seq Scan on participation p  (cost=0.00..164.00 rows=10000 width=16)"
"        ->  Hash  (cost=279.00..279.00 rows=50 width=8)"
"              ->  Seq Scan on event e  (cost=0.00..279.00 rows=50 width=8)"
"                    Filter: (eventname = 'event1'::text)"
"  ->  Index Scan using user_pkey on "user" u  (cost=0.00..0.27 rows=1 width=108)"
"        Index Cond: (userid = p.userid)"
\end{lstlisting}

sowie Indexen f�r die Fremdschl�ssel:
\begin{lstlisting}[caption=Indexe f�r alle Fremdschl�ssel setzen, firstnumber=1, label=PrimaerUndFremdschluesselUndIndex]{code:PrimaerUndFremdschluesselUndIndex}
CREATE INDEX event_creatorid ON public.event(creatorid);
CREATE INDEX message_eventid ON public.message(eventid);
CREATE INDEX message_senderid ON public.message(senderid);
CREATE INDEX message_recipientid ON public.message(recipientid);
CREATE INDEX participation_userid ON public.participation(userid);
CREATE INDEX participation_eventid ON public.participation(eventid);
\end{lstlisting}

\begin{lstlisting}[caption=Ausf�hrungsplan Query 1 mit Index f�r alle Fremdschl�ssel, firstnumber=1, label=aQuery1PrimaerUndFremdschluesselUndIndexFremdschluessel]{code:aQuery1PrimaerUndFremdschluesselUndIndexFremdschluessel}
"Nested Loop  (cost=0.00..287.57 rows=1 width=38)"
"  ->  Nested Loop  (cost=0.00..287.28 rows=1 width=8)"
"        ->  Seq Scan on event e  (cost=0.00..279.00 rows=1 width=8)"
"              Filter: (eventname = 'event1'::text)"
"        ->  Index Scan using participation_eventid on participation p  (cost=0.00..8.27 rows=1 width=16)"
"              Index Cond: (eventid = e.eventid)"
"  ->  Index Scan using user_pkey on "user" u  (cost=0.00..0.27 rows=1 width=38)"
"        Index Cond: (userid = p.userid)"
\end{lstlisting}

Indexe auf Nicht-Id-Spalten
\begin{lstlisting}[caption=Indexe f�r alle Fremdschl�ssel setzen, firstnumber=1, label=PrimaerUndFremdschluesselUndIndexNichtId]{code:PrimaerUndFremdschluesselUndIndexNichtId}
CREATE INDEX event_date ON public.event(date);
CREATE INDEX event_eventname ON public.event(eventname);
CREATE INDEX event_occasion ON public.event(occasion);
CREATE INDEX event_location ON public.event(location);
CREATE INDEX event_lon ON public.event(lon);
CREATE INDEX event_lat ON public.event(lat);
CREATE INDEX event_numbermaleconfirmed ON public.event(numbermaleconfirmed);
CREATE INDEX event_numberfemaleconfirmed ON public.event(numberfemaleconfirmed);
CREATE INDEX message_textmessage ON public.message(textmessage);
CREATE INDEX message_date ON public.message(date);
CREATE INDEX user_name ON public.user(name);
CREATE INDEX user_email ON public.user(email);
CREATE INDEX user_gender ON public.user(gender);
CREATE INDEX user_birthday ON public.user(birthday);
\end{lstlisting}

\begin{lstlisting}[caption=Ausf�hrungsplan Query 1 mit Indexen f�r alle Fremdschl�ssel, firstnumber=1, label=aQuery1PrimaerUndFremdschluesselUndIndexFremdschluesselNichtId]{code:aQuery1PrimaerUndFremdschluesselUndIndexFremdschluesselNichtId}
"Nested Loop  (cost=0.00..16.84 rows=1 width=38)"
"  ->  Nested Loop  (cost=0.00..16.55 rows=1 width=8)"
"        ->  Index Scan using event_eventname on event e  (cost=0.00..8.27 rows=1 width=8)"
"              Index Cond: (eventname = 'event1'::text)"
"        ->  Index Scan using participation_eventid on participation p  (cost=0.00..8.27 rows=1 width=16)"
"              Index Cond: (eventid = e.eventid)"
"  ->  Index Scan using user_pkey on "user" u  (cost=0.00..0.27 rows=1 width=38)"
"        Index Cond: (userid = p.userid)"
\end{lstlisting}
-------------------------------------------------------------

Query 2: Alle weiblichen user anzeigen, die zwischen 1986 und 1992 geboren worden und an einem event teilnehmen, dass  zwischen dem 01.01.2013 und dem 01.03.2013 stattfindet, und bei dem numberMaleConfirmed / numberFemaleConfirmed kleiner 0,5 ist.

\begin{lstlisting}[caption=Query 2, firstnumber=1, label=Query2]{code:Query2}
Select u.userId, u.name, u.email, u.gender, u.birthday
from public.user u, public.event e, public.participation p
where u.gender = 'w'
AND u.birthday between '01.01.1986' AND '31.12.1992'
AND e.date between '01.01.2013' AND '01.03.2013'
AND e.eventid = p.eventid
AND p.userid = u.userid
AND e.numberMaleConfirmed / e.numberFemaleConfirmed < 0.5;
\end{lstlisting}

\begin{lstlisting}[caption=Ausf�hrungsplan f�r Query 2, firstnumber=1, label=aQuery2]{code:aQuery2}
"Hash Join  (cost=218.37..511.08 rows=2 width=108)"
"  Hash Cond: (e.eventid = p.eventid)"
"  ->  Seq Scan on event e  (cost=0.00..292.60 rows=10 width=8)"
"        Filter: ((date >= '2013-01-01'::date) AND (date <= '2013-03-01'::date) AND (((numbermaleconfirmed / numberfemaleconfirmed))::numeric < 0.5))"
"  ->  Hash  (cost=217.77..217.77 rows=48 width=116)"
"        ->  Hash Join  (cost=20.41..217.77 rows=48 width=116)"
"              Hash Cond: (p.userid = u.userid)"
"              ->  Seq Scan on participation p  (cost=0.00..160.64 rows=9664 width=16)"
"              ->  Hash  (cost=20.40..20.40 rows=1 width=108)"
"                    ->  Seq Scan on "user" u  (cost=0.00..20.40 rows=1 width=108)"
"                          Filter: ((birthday >= '1986-01-01'::date) AND (birthday <= '1992-12-31'::date) AND (gender = 'w'::text))"
\end{lstlisting}


mit Prim�r- und Fremdschl�sseln f�r alle Tabellen:

\begin{lstlisting}[caption=Ausf�hrungsplan Query 2 mit Prim�r- und Fremdschl�sseln, firstnumber=1, label=aQuery2PrimaerUndFremdschluessel]{code:aQuery2PrimaerUndFremdschluessel}
"Nested Loop  (cost=29.96..371.06 rows=10 width=38)"
"  ->  Hash Join  (cost=29.96..235.16 rows=370 width=46)"
"        Hash Cond: (p.userid = u.userid)"
"        ->  Seq Scan on participation p  (cost=0.00..164.00 rows=10000 width=16)"
"        ->  Hash  (cost=29.50..29.50 rows=37 width=38)"
"              ->  Seq Scan on "user" u  (cost=0.00..29.50 rows=37 width=38)"
"                    Filter: ((birthday >= '1986-01-01'::date) AND (birthday <= '1992-12-31'::date) AND (gender = 'w'::text))"
"  ->  Index Scan using event_pkey on event e  (cost=0.00..0.35 rows=1 width=8)"
"        Index Cond: (eventid = p.eventid)"
"        Filter: ((date >= '2013-01-01'::date) AND (date <= '2013-03-01'::date) AND (((numbermaleconfirmed / numberfemaleconfirmed))::numeric < 0.5))"
\end{lstlisting}

sowie Indexen auf Fremdschl�ssel- und Nicht-Id-Spalten:

\begin{lstlisting}[caption=Ausf�hrungsplan Query 2 mit Indexen, firstnumber=1, label=aQuery2PrimaerUndFremdschluesselUndIndexFremdschluesselNichtId]{code:aQuery2PrimaerUndFremdschluesselUndIndexFremdschluesselNichtId}
"Nested Loop  (cost=18.70..359.80 rows=10 width=38)"
"  ->  Hash Join  (cost=18.70..223.90 rows=370 width=46)"
"        Hash Cond: (p.userid = u.userid)"
"        ->  Seq Scan on participation p  (cost=0.00..164.00 rows=10000 width=16)"
"        ->  Hash  (cost=18.24..18.24 rows=37 width=38)"
"              ->  Bitmap Heap Scan on "user" u  (cost=4.98..18.24 rows=37 width=38)"
"                    Recheck Cond: ((birthday >= '1986-01-01'::date) AND (birthday <= '1992-12-31'::date))"
"                    Filter: (gender = 'w'::text)"
"                    ->  Bitmap Index Scan on user_birthday  (cost=0.00..4.97 rows=72 width=0)"
"                          Index Cond: ((birthday >= '1986-01-01'::date) AND (birthday <= '1992-12-31'::date))"
"  ->  Index Scan using event_pkey on event e  (cost=0.00..0.35 rows=1 width=8)"
"        Index Cond: (eventid = p.eventid)"
"        Filter: ((date >= '2013-01-01'::date) AND (date <= '2013-03-01'::date) AND (((numbermaleconfirmed / numberfemaleconfirmed))::numeric < 0.5))"
\end{lstlisting}

-------------------------------------------------------------

Query 3: Die Anzahl der textmessages gruppiert und absteigend sortiert nach numberFemaleConfirmed, die zwischen dem 01.01.2010 und dem 31.12.2012 geschrieben wurden, in denen das Wort Salsa vorkommt, deren Empf�nger m�nnlich sind und zwischen 1972 und 1982 geboren wurden, deren Absender weiblich sind und die zwischen 1986 und 1992 geboren worden und an einem event teilnehmen, dass  zwischen dem 01.01.2013 und dem 01.03.2013 in einem 100km Radius zu der GPS-Koordinate 11.5833 45.15.00 stattfindet und bei dem numberMaleConfirmed / numberFemaleConfirmed kleiner 0,5 ist. Das Ergebnis soll nur die ersten f�nf Treffer liefern.

\begin{lstlisting}[caption=Query 3, firstnumber=1, label=Query3]{code:Query3}
Select e.numberFemaleConfirmed, Count(m.messageid) as anzahlMessages
from public.user u, public.event e, public.participation p, public.message m
where u.gender = 'w'
AND u.birthday between '01.01.1986' AND '31.12.1992'
AND e.date between '01.01.2013' AND '01.03.2013'
AND m.date between '01.01.2010' AND '31.12.2012'
AND m.textmessage like '%Salsa'
AND e.eventid = p.eventid
AND p.userid = u.userid
AND m.senderId = u.userId
AND e.numberMaleConfirmed / e.numberFemaleConfirmed < 0.5
AND DEGREES(acos(cos(RADIANS(90-e.lat))*cos(RADIANS(90-70))+sin(RADIANS(90-e.lat))*
sin(RADIANS(90-70))*cos(RADIANS(e.lon-80))))/360*40074 < 10000
AND m.recipientId in
(Select u2.userId
from public.user u2
where u2.gender = 'm'
AND u2.birthday between '01.01.1972' AND '31.12.1982'
)
Group by e.numberFemaleConfirmed
Order by e.numberFemaleConfirmed DESC
Limit 5
\end{lstlisting}

\begin{lstlisting}[caption=Ausf�hrungsplan f�r Query 3, firstnumber=1, label=aQuery3]{code:aQuery3}
"Limit  (cost=1386.84..1386.86 rows=1 width=12)"
"  ->  GroupAggregate  (cost=1386.84..1386.86 rows=1 width=12)"
"        ->  Sort  (cost=1386.84..1386.84 rows=1 width=12)"
"              Sort Key: e.numberfemaleconfirmed"
"              ->  Nested Loop  (cost=269.91..1386.83 rows=1 width=12)"
"                    Join Filter: (m.senderid = p.userid)"
"                    ->  Hash Join  (cost=239.79..1069.16 rows=3 width=20)"
"                          Hash Cond: (e.eventid = p.eventid)"
"                          ->  Seq Scan on event e  (cost=0.00..829.00 rows=92 width=12)"
"                                Filter: ((date >= '2013-01-01'::date) AND (date <= '2013-03-01'::date) AND (((numbermaleconfirmed / numberfemaleconfirmed))::numeric < 0.5) AND (((degrees(acos(((cos(radians((90::double precision - lat))) * 0.939692620785908::double precision) + ((sin(radians((90::double precision - lat))) * 0.342020143325669::double precision) * cos(radians((lon - 80::double precision))))))) / 360::double precision) * 40074::double precision) < 10000::double precision))"
"                          ->  Hash  (cost=235.16..235.16 rows=370 width=24)"
"                                ->  Hash Join  (cost=29.96..235.16 rows=370 width=24)"
"                                      Hash Cond: (p.userid = u.userid)"
"                                      ->  Seq Scan on participation p  (cost=0.00..164.00 rows=10000 width=16)"
"                                      ->  Hash  (cost=29.50..29.50 rows=37 width=8)"
"                                            ->  Seq Scan on "user" u  (cost=0.00..29.50 rows=37 width=8)"
"                                                  Filter: ((birthday >= '1986-01-01'::date) AND (birthday <= '1992-12-31'::date) AND (gender = 'w'::text))"
"                    ->  Materialize  (cost=30.13..314.39 rows=77 width=16)"
"                          ->  Hash Semi Join  (cost=30.13..314.01 rows=77 width=16)"
"                                Hash Cond: (m.recipientid = u2.userid)"
"                                ->  Seq Scan on message m  (cost=0.00..279.00 rows=1533 width=24)"
"                                      Filter: ((date >= '2010-01-01'::date) AND (date <= '2012-12-31'::date) AND (textmessage ~~ '%Salsa'::text))"
"                                ->  Hash  (cost=29.50..29.50 rows=50 width=8)"
"                                      ->  Seq Scan on "user" u2  (cost=0.00..29.50 rows=50 width=8)"
"                                            Filter: ((birthday >= '1972-01-01'::date) AND (birthday <= '1982-12-31'::date) AND (gender = 'm'::text))"
\end{lstlisting}

\begin{lstlisting}[caption=Ausf�hrungsplan Query 3 mit Prim�r- und Fremdschl�sseln, firstnumber=1, label=aQuery3PrimaerUndFremdschluessel]{code:aQuery3PrimaerUndFremdschluessel}
mit Prim�r- und Fremdschl�sseln f�r alle Tabellen:
"Limit  (cost=546.93..546.95 rows=1 width=12)"
"  ->  GroupAggregate  (cost=546.93..546.95 rows=1 width=12)"
"        ->  Sort  (cost=546.93..546.93 rows=1 width=12)"
"              Sort Key: e.numberfemaleconfirmed"
"              ->  Nested Loop  (cost=333.57..546.92 rows=1 width=12)"
"                    ->  Hash Join  (cost=333.57..535.37 rows=28 width=16)"
"                          Hash Cond: (p.userid = u.userid)"
"                          ->  Seq Scan on participation p  (cost=0.00..164.00 rows=10000 width=16)"
"                          ->  Hash  (cost=333.53..333.53 rows=3 width=24)"
"                                ->  Nested Loop Semi Join  (cost=29.96..333.53 rows=3 width=24)"
"                                      ->  Hash Join  (cost=29.96..315.28 rows=57 width=32)"
"                                            Hash Cond: (m.senderid = u.userid)"
"                                            ->  Seq Scan on message m  (cost=0.00..279.00 rows=1533 width=24)"
"                                                  Filter: ((date >= '2010-01-01'::date) AND (date <= '2012-12-31'::date) AND (textmessage ~~ '%Salsa'::text))"
"                                            ->  Hash  (cost=29.50..29.50 rows=37 width=8)"
"                                                  ->  Seq Scan on "user" u  (cost=0.00..29.50 rows=37 width=8)"
"                                                        Filter: ((birthday >= '1986-01-01'::date) AND (birthday <= '1992-12-31'::date) AND (gender = 'w'::text))"
"                                      ->  Index Scan using user_pkey on "user" u2  (cost=0.00..0.32 rows=1 width=8)"
"                                            Index Cond: (userid = m.recipientid)"
"                                            Filter: ((birthday >= '1972-01-01'::date) AND (birthday <= '1982-12-31'::date) AND (gender = 'm'::text))"
"                    ->  Index Scan using event_pkey on event e  (cost=0.00..0.40 rows=1 width=12)"
"                          Index Cond: (eventid = p.eventid)"
"                          Filter: ((date >= '2013-01-01'::date) AND (date <= '2013-03-01'::date) AND (((numbermaleconfirmed / numberfemaleconfirmed))::numeric < 0.5) AND (((degrees(acos(((cos(radians((90::double precision - lat))) * 0.939692620785908::double precision) + ((sin(radians((90::double precision - lat))) * 0.342020143325669::double precision) * cos(radians((lon - 80::double precision))))))) / 360::double precision) * 40074::double precision) < 10000::double precision))"
\end{lstlisting}


sowie Indexen auf Fremdschl�ssel- und Nicht-Id-Spalten:

\begin{lstlisting}[caption=Ausf�hrungsplan Query 3 mit Indexen, firstnumber=1, label=aQuery3PrimaerUndFremdschluesselUndIndexFremdschluesselNichtId]{code:aQuery3PrimaerUndFremdschluesselUndIndexFremdschluesselNichtId}
"Limit  (cost=355.27..355.29 rows=1 width=12)"
"  ->  GroupAggregate  (cost=355.27..355.29 rows=1 width=12)"
"        ->  Sort  (cost=355.27..355.28 rows=1 width=12)"
"              Sort Key: e.numberfemaleconfirmed"
"              ->  Nested Loop  (cost=18.70..355.26 rows=1 width=12)"
"                    ->  Nested Loop  (cost=18.70..343.72 rows=28 width=16)"
"                          ->  Nested Loop Semi Join  (cost=18.70..322.27 rows=3 width=24)"
"                                ->  Hash Join  (cost=18.70..304.02 rows=57 width=32)"
"                                      Hash Cond: (m.senderid = u.userid)"
"                                      ->  Seq Scan on message m  (cost=0.00..279.00 rows=1533 width=24)"
"                                            Filter: ((date >= '2010-01-01'::date) AND (date <= '2012-12-31'::date) AND (textmessage ~~ '%Salsa'::text))"
"                                      ->  Hash  (cost=18.24..18.24 rows=37 width=8)"
"                                            ->  Bitmap Heap Scan on "user" u  (cost=4.98..18.24 rows=37 width=8)"
"                                                  Recheck Cond: ((birthday >= '1986-01-01'::date) AND (birthday <= '1992-12-31'::date))"
"                                                  Filter: (gender = 'w'::text)"
"                                                  ->  Bitmap Index Scan on user_birthday  (cost=0.00..4.97 rows=72 width=0)"
"                                                        Index Cond: ((birthday >= '1986-01-01'::date) AND (birthday <= '1992-12-31'::date))"
"                                ->  Index Scan using user_pkey on "user" u2  (cost=0.00..0.32 rows=1 width=8)"
"                                      Index Cond: (userid = m.recipientid)"
"                                      Filter: ((birthday >= '1972-01-01'::date) AND (birthday <= '1982-12-31'::date) AND (gender = 'm'::text))"
"                          ->  Index Scan using participation_userid on participation p  (cost=0.00..7.02 rows=10 width=16)"
"                                Index Cond: (userid = u.userid)"
"                    ->  Index Scan using event_pkey on event e  (cost=0.00..0.40 rows=1 width=12)"
"                          Index Cond: (eventid = p.eventid)"
"                          Filter: ((date >= '2013-01-01'::date) AND (date <= '2013-03-01'::date) AND (((numbermaleconfirmed / numberfemaleconfirmed))::numeric < 0.5) AND (((degrees(acos(((cos(radians((90::double precision - lat))) * 0.939692620785908::double precision) + ((sin(radians((90::double precision - lat))) * 0.342020143325669::double precision) * cos(radians((lon - 80::double precision))))))) / 360::double precision) * 40074::double precision) < 10000::double precision))"
\end{lstlisting}
\section{Was tun bei langsamen Ausf�hrungspl�nen?}\label{WasTunBeiLangsamenAusfuehrungsplaenen}

\begin{itemize}
\item Um dem Planer keine falschen Hinweise zu geben, sollten keine Hints verwendet werden.

\item Autoanalyze aktivieren\\
Der Planner kann nur dann optimierte Ausf�hrungspl�ne erzeugen, wenn er gen�gend Statistiken �ber die gef�llten Tabellen besitzt. Deswegen m�ssen regelm��ig Statistiken erstellt werden. In PostgreSQL geht das mit analyze. Analyze sammelt Informationen �ber den F�llstand der Tabellen, die h�ufigsten Werte in jeder Spalten und die wahrscheinliche Verteilung der Werte in einer Spalte.
Mit diesen Statistiken kann der Planer dann den passenden JOIN-Algorithmus und die passende JOIN-Order w�hlen.
Um nach bestimmten Abst�nden automatisch analyze aufzurufen, gibt es autoanalyze.

\item Autovacuum aktivieren \\
Mit autovacuum = on in der postgresql.conf kann autovacuum aktiviert werden.

\item Indexe verwenden \\
Es kann z.B. der Entwickler geeignete Indexe anlegen, sodass das DBMS diese verwenden kann, um performantere Ausf�hrungspl�ne zu erzeugen.

\item Mehrspaltige Indexe verwenden, wenn dadurch ein performanterer 
Aus\-f�hr\-ungs\-plan erzeugt werden kann.

\item Keine zu intensiven Rechnungen in SQL formulieren.

\item Partitioning verwenden.

\item Die Struktur der Tabellen �berdenken, wenn mehr als 8 Tabellen miteinander verkn�pft werden.

\item Herausfinden, warum der Planner einen langsamen Plan erzeugt, anstatt durch Planner-Hints der Frage aus dem Weg zu gehen.

\item Es kann sein, dass jemand ung�nstige Werte f�r die Parameter in der postgresql.conf gesetzt hat.

\item Bei einem zu kleinen Wert f�r $join\_collapse\_limit$ in der postgresql.conf, verbunden mit der expliziten Verkn�pfung von Tabellen mit dem Wort Join, ist der Planner gezwungen, eine vorgegebene aber u.U. ung�nstige Join-Reihenfolge zu verwenden.

\item Um dem Planner nicht ausversehen zu etwas zu zwingen, sollten bei einem Innerjoin die Tabellen nicht explizit mit dem Wort JOIN verkn�pfen werden, sondern es sollten die Tabellen einfach getrennt durch ein Komma angeben werden:

\begin{lstlisting}[caption=Inner-Join, firstnumber=1, label=gtlt]{code:gtlt}
SELECT * FROM a, b, c WHERE a.id = b.id AND b.ref = c.id;
\end{lstlisting}

\item Bei einem zu kleinen Wert f�r $from\_collapse\_limit$ in der postgresql.conf, verbunden mit vielen Subqueries, kann der Planner die Subqueries nicht aufl�sen. Dabei sollte der Planner Subqueries zu Joins aufl�sen, da sonst erst das komplette Ergebnis des Subquerys erstellt werden muss, bevor mit der Tabelle weitergearbeitet werden kann.
\end{itemize}
\section{Zusammenfassung}\label{Zusammenfassung}
Der Planer von PostgreSQL ist so ausgereift, dass dessen Entwickler bewusst auf ein Hints-System verzichten. Es wird davon ausgegangen, dass der Planer umso bessere Ausf�hrungspl�ne macht, je mehr Entscheidungsfreiheit er bei der Planerstellung hat. Diese Entscheidungsfreiheit w�rden durch Hints eingeschr�nkt. Ausf�hrungsplanoptimierung beschr�nkt sich bei PostgreSQL vor allem darauf, sicherzustellen, dass der Planer immer aktuelle Statistiken hat und ihm m�glichst viele Indexe zur Verf�gung stehen, aus denen er die seiner meiner nach Besten w�hlen kann, um einen effizienten Ausf�hrungsplan zu erstellen.
\section{Ausblick}\label{Ausblick}
In einem weiteren Projekt k�nnte man die Ausf�hrungspl�ne bei Tabellen mit Partitioning untersuchen.

Es k�nnte auch die Arbeitsweise des Planers untersucht werden, wenn mehr als f�nfzehn Tabellen miteinander verkn�pft werden und er mit einer genetischen Suche nach einem effizienten Auf�hrungsplan sucht.

Es k�nnte gezeigt werden, welchen Einfluss der F�llstand einer Tabelle und die Verteilung der Werte in einer einzelnen Spalte auf die Generierung eines Ausf�hrungsplanes hat.


%\section{Die Abarbeitung von Abfragen in PostgreSQL}
\begin{enumerate} 
\item \textbf{Empfang des SQL-Befehls} \\
Nachdem der SQL-Befehl �ber eine Netzwerkverbindung �bertragen wurde, findet die Kodierungsumwandlung statt, 
und die weiteren Phasen der Abarbeitung sehen den Befehl in der Serverkodierung. Hierbei gibt es nur sehr
geringe Optimierungsm�glichkeiten. Es k�nnen theoretisch CPU-Zyklen gespart werden, wenn die Clientkodierung 
gleich der Serverkodierung ist, ansonsten wird eine Konvertierung durchgef�hrt. Diese Auswirkungen sind jedoch sehr gering.
Der Parameter \textit{client\_encoding} informiert den Server dar�ber, welche Kodierung die ankommenden Befehle haben 
und welche Kodierung das Anfrageergebnis haben soll, welches an den Client gesendet wird. Die Voreinstellung 
gibt an welche Kodierung der Server intern verwendet.

\item \textbf{Parser} \\
In dieser Abarbeitungsphase wird die kodierte Befehlszeichenkette durch einen internen Parse-Baum dargestellt. 
Des Weiteren wird die Befehlszeichenkette auf semantische Bedingungen �berpr�ft und etwas bearbeitet. Die SQL-Befehle werden dann 
aufgeteilt in sogenannte optimierbare Anweisungen(SELECT, INSERT, UPDATE und DELETE) und Hilfsanweisungen.
Die Hilfsanweisungen werden sp�ter direkt ausgef�hrt und sie erzeugen keine Ausgabe. Dagegen kommen die optimierbaren 
Anweisungen in den Rewriter. F�r den Parser gibt es von der Anwenderseite keine M�glichkeit die Geschwindigkeit 
zu optimieren. 


\item \textbf{Query Rewriter} \\
Der Rewriter wendet die Anfrageumschreibregeln(Query Rewrite Rules) an. Dabei werden die Sichten(Views) und andere
benutzerdefinierte Regeln aufgel�st, in die Anfrage eingebaut und im Parse-Baum ersetzt. Da der Rewriter 
vor dem Planer angesiedelt ist, bekommt der Planer es nicht mit, ob die Anfrage aus einer Sicht kam oder nicht. Mit der 
Erstellung einer Sicht hat man somit keinen Optimierungsvorteil.

\item \textbf{Planer / Optimizer} \\
Der Planer bekommt den m�glicherweise umgeschriebenen Parse-Baum und hat die Aufgabe einen Ausf�hrungsplan(execution plan) zu erstellen, der ebenfalls ein Baum ist.
Der Ausf�hrungsplan beschreibt wie auf die Tabellen zugegriffen werden soll, also welche Indexe und Join-Algorithmen verwendet werden sollen und in welcher Reihenfolge. Es soll m�glichst der optimalste und schnellste Ausf�hrungsplan gefunden werden. 

\item \textbf{Executor} \\ 
Der vom Planer auserw�hlte Ausf�hrungsplan wird vom Executor ausgef�hrt. Dabei werden Zugriffsrechte auf Tabellen und andere Objekte sowie Constraints gepr�ft. Die Laufzeit der Ausf�hrung h�ngt nicht nur davon ab ob der Plan gut ist, sondern auch von der gesamten Systemkonfiguration.

\end{enumerate}


%\chapter{Theoretische Grundlagen}
%Die f\"{u}r den Untersuchungsgegenstand relevanten Themen, die \"{u}ber die
%grundlegenden Studieninhalte hinausgehen; oft auch anwendungsspezifische Aspekte - %ca. 6 Seiten

%\chapter{Ist-Analyse}
%Welche Defizite sollen mit der Arbeit behoben werden, welche nicht? %Pr\"{a}zisierung
%der Zielstellung - ca. 6 Seiten

%\chapter{L\"{o}sungskonzept}
%Wie sollen die Defizite behoben werden? Methoden, fachliche Auseinandersetzung
%mit alternativen Ans\"{a}tzen und Auffassungen, Systembeschreibung (Architektur,
%Vorgehensmodell, \ldots) - ca. 12 Seiten

%\chapter{Implementierung}
%Umsetzung des L\"{o}sungskonzepts, Begr\"{u}ndung der verwendeten Technologien - %ca. 8
%Seiten

%\chapter{Ergebnisse}
%Objektive Bewertung der vorliegenden L\"{o}sung, diverse Testverfahren,
%Nutzerbefragungen - ca. 4 Seiten

%\chapter{Fazit und Ausblick}
%Zusammenfassung s\"{a}mtlicher Ergebnisse in Bezug auf die Zielerf\"{u}llung und
%Vorschl\"{a}ge f\"{u}r weiterf\"{u}hrende Arbeiten - ca. 2 Seiten

\bibliographystyle{alphadin}
\begin{appendix}
\newpage
\pagestyle{appendixAStyle}
\chapter{Codebeispiele}
\begin{lstlisting}[caption=Testabfrage 1, firstnumber=1]{code:uc1}
Select adbc.Produkt.Name, Count(*)
From adbc.Produkt, adbc.Warenkorb_has_Produkt
where adbc.Produkt.PRODUKT_ID = adbc.Warenkorb_has_Produkt.Produkt_PRODUKT_ID
Group by adbc.Produkt.Name;
\end{lstlisting}

\begin{lstlisting}[caption=Testabfrage 2, firstnumber=1]{code:uc2}
SELECT adbc.kunde.Name, SUM(adbc.produkt.Preis)
FROM adbc.kunde,adbc.Produkt, adbc.Warenkorb, adbc.Warenkorb_has_Produkt
WHERE produkt.PRODUKT_ID = warenkorb_has_produkt.Produkt_PRODUKT_ID 
    AND warenkorb_has_produkt.Warenkorb_WARENKORB_ID = warenkorb.WARENKORB_ID
    AND warenkorb.Kunde_KUNDE_ID = kunde.KUNDE_ID    
    AND (warenkorb_has_produkt.Datum BETWEEN '2011-01-01' AND '2011-03-01')
Group by adbc.kunde.Name;
\end{lstlisting}

\begin{lstlisting}[caption=Testabfrage 3, firstnumber=1]{code:uc3}
SELECT Count(DISTINCT(adbc.kunde.KUNDE_ID))
FROM adbc.Kunde, adbc.Warenkorb, adbc.warenkorb_has_produkt
WHERE warenkorb.Kunde_KUNDE_ID = kunde.KUNDE_ID
AND warenkorb.WARENKORB_ID = warenkorb_has_produkt.Warenkorb_WARENKORB_ID
    AND warenkorb_has_produkt.Datum = '2011-01-01';
\end{lstlisting}

\begin{lstlisting}[caption=Testabfrage 4, firstnumber=1]{code:uc4}
SELECT adbc.kunde.Name, SUM(adbc.produkt.Preis)
FROM adbc.kunde,adbc.Produkt, adbc.Warenkorb, adbc.Warenkorb_has_Produkt
WHERE produkt.PRODUKT_ID = warenkorb_has_produkt.Produkt_PRODUKT_ID 
    AND warenkorb_has_produkt.Warenkorb_WARENKORB_ID = warenkorb.WARENKORB_ID
    AND warenkorb.Kunde_KUNDE_ID = kunde.KUNDE_ID
    AND ((warenkorb_has_produkt.Datum = '2011-01-01')
    OR   (warenkorb_has_produkt.Datum = '2011-01-05')
    OR   (warenkorb_has_produkt.Datum = '2011-01-09')    
    OR   (warenkorb_has_produkt.Datum = '2011-02-02')
    OR   (warenkorb_has_produkt.Datum = '2011-02-06')    
    OR   (warenkorb_has_produkt.Datum = '2011-02-10')
    OR   (warenkorb_has_produkt.Datum = '2011-03-10')
    OR   (warenkorb_has_produkt.Datum = '2011-03-14')    
    OR   (warenkorb_has_produkt.Datum = '2011-03-18'))      
Group by adbc.kunde.Name;
\end{lstlisting}

\begin{lstlisting}[caption=Tabellenerzeugung mit Hash-Partitioning, firstnumber=1]{code:createparthash}
SET @OLD_UNIQUE_CHECKS=@@UNIQUE_CHECKS, UNIQUE_CHECKS=0;
SET @OLD_FOREIGN_KEY_CHECKS=@@FOREIGN_KEY_CHECKS, FOREIGN_KEY_CHECKS=0;
SET @OLD_SQL_MODE=@@SQL_MODE, SQL_MODE='TRADITIONAL';

CREATE SCHEMA IF NOT EXISTS `ADBC` DEFAULT CHARACTER SET utf8 COLLATE utf8_general_ci ;

USE `ADBC`;

CREATE  TABLE IF NOT EXISTS `ADBC`.`Kunde` (
  `KUNDE_ID` INT(11) NULL DEFAULT NULL ,
  `Name` VARCHAR(45) NULL DEFAULT NULL ,
  `Kundennummer` VARCHAR(45) NULL DEFAULT NULL )
ENGINE = InnoDB
DEFAULT CHARACTER SET = utf8
COLLATE = utf8_general_ci;

CREATE  TABLE IF NOT EXISTS `ADBC`.`Warenkorb` (
  `WARENKORB_ID` INT(11) NULL DEFAULT NULL ,
  `Kunde_KUNDE_ID` INT(11) NULL DEFAULT NULL )
ENGINE = InnoDB
DEFAULT CHARACTER SET = utf8
COLLATE = utf8_general_ci;

CREATE  TABLE IF NOT EXISTS `ADBC`.`Produkt` (
  `PRODUKT_ID` INT(11) NULL DEFAULT NULL ,
  `Name` VARCHAR(45) NULL DEFAULT NULL ,
  `Preis` INT(11) NULL DEFAULT NULL )
ENGINE = InnoDB
DEFAULT CHARACTER SET = utf8
COLLATE = utf8_general_ci;

CREATE  TABLE IF NOT EXISTS `ADBC`.`Warenkorb_has_Produkt` (
  `Warenkorb_WARENKORB_ID` INT(11) NULL DEFAULT NULL ,
  `Produkt_PRODUKT_ID` INT(11) NULL DEFAULT NULL ,
  `WARENKORB_HAS_PRODUKT_ID` INT(11) NULL DEFAULT NULL ,
  `Datum` DATE NULL DEFAULT NULL )
ENGINE = InnoDB
DEFAULT CHARACTER SET = utf8
COLLATE = utf8_general_ci

    PARTITION BY HASH( MONTH(Datum) )
    PARTITIONS 12;

;


SET SQL_MODE=@OLD_SQL_MODE;
SET FOREIGN_KEY_CHECKS=@OLD_FOREIGN_KEY_CHECKS;
SET UNIQUE_CHECKS=@OLD_UNIQUE_CHECKS;
\end{lstlisting}

\begin{lstlisting}[caption=Tabellenerzeugung mit List-Partitioning, firstnumber=1]{code:createpartlist}
SET @OLD_UNIQUE_CHECKS=@@UNIQUE_CHECKS, UNIQUE_CHECKS=0;
SET @OLD_FOREIGN_KEY_CHECKS=@@FOREIGN_KEY_CHECKS, FOREIGN_KEY_CHECKS=0;
SET @OLD_SQL_MODE=@@SQL_MODE, SQL_MODE='TRADITIONAL';

CREATE SCHEMA IF NOT EXISTS `ADBC` DEFAULT CHARACTER SET utf8 COLLATE utf8_general_ci ;

USE `ADBC`;

CREATE  TABLE IF NOT EXISTS `ADBC`.`Kunde` (
  `KUNDE_ID` INT(11) NULL DEFAULT NULL ,
  `Name` VARCHAR(45) NULL DEFAULT NULL ,
  `Kundennummer` VARCHAR(45) NULL DEFAULT NULL )
ENGINE = InnoDB
DEFAULT CHARACTER SET = utf8
COLLATE = utf8_general_ci;

CREATE  TABLE IF NOT EXISTS `ADBC`.`Warenkorb` (
  `WARENKORB_ID` INT(11) NULL DEFAULT NULL ,
  `Kunde_KUNDE_ID` INT(11) NULL DEFAULT NULL )
ENGINE = InnoDB
DEFAULT CHARACTER SET = utf8
COLLATE = utf8_general_ci;

CREATE  TABLE IF NOT EXISTS `ADBC`.`Produkt` (
  `PRODUKT_ID` INT(11) NULL DEFAULT NULL ,
  `Name` VARCHAR(45) NULL DEFAULT NULL ,
  `Preis` INT(11) NULL DEFAULT NULL )
ENGINE = InnoDB
DEFAULT CHARACTER SET = utf8
COLLATE = utf8_general_ci;

CREATE  TABLE IF NOT EXISTS `ADBC`.`Warenkorb_has_Produkt` (
  `Warenkorb_WARENKORB_ID` INT(11) NULL DEFAULT NULL ,
  `Produkt_PRODUKT_ID` INT(11) NULL DEFAULT NULL ,
  `WARENKORB_HAS_PRODUKT_ID` INT(11) NULL DEFAULT NULL ,
  `Datum` DATE NULL DEFAULT NULL )
    
ENGINE = InnoDB
DEFAULT CHARACTER SET = utf8
COLLATE = utf8_general_ci

PARTITION BY LIST(MONTH(Datum)) (
    PARTITION quartal1 VALUES IN (1,2,3),
    PARTITION quartal2 VALUES IN (4,5,6),
    PARTITION quartal3 VALUES IN (7,8,9),
    PARTITION quartal4 VALUES IN (10,11,12)
);


SET SQL_MODE=@OLD_SQL_MODE;
SET FOREIGN_KEY_CHECKS=@OLD_FOREIGN_KEY_CHECKS;
SET UNIQUE_CHECKS=@OLD_UNIQUE_CHECKS;

\end{lstlisting}

\begin{lstlisting}[caption=Tabellenerzeugung mit Range-Partitioning, firstnumber=1]{code:createpartrange}
SET @OLD_UNIQUE_CHECKS=@@UNIQUE_CHECKS, UNIQUE_CHECKS=0;
SET @OLD_FOREIGN_KEY_CHECKS=@@FOREIGN_KEY_CHECKS, FOREIGN_KEY_CHECKS=0;
SET @OLD_SQL_MODE=@@SQL_MODE, SQL_MODE='TRADITIONAL';

CREATE SCHEMA IF NOT EXISTS `ADBC` DEFAULT CHARACTER SET utf8 COLLATE utf8_general_ci ;

USE `ADBC`;

CREATE  TABLE IF NOT EXISTS `ADBC`.`Kunde` (
  `KUNDE_ID` INT(11) NULL DEFAULT NULL ,
  `Name` VARCHAR(45) NULL DEFAULT NULL ,
  `Kundennummer` VARCHAR(45) NULL DEFAULT NULL )
ENGINE = InnoDB
DEFAULT CHARACTER SET = utf8
COLLATE = utf8_general_ci;

CREATE  TABLE IF NOT EXISTS `ADBC`.`Warenkorb` (
  `WARENKORB_ID` INT(11) NULL DEFAULT NULL ,
  `Kunde_KUNDE_ID` INT(11) NULL DEFAULT NULL )
ENGINE = InnoDB
DEFAULT CHARACTER SET = utf8
COLLATE = utf8_general_ci;

CREATE  TABLE IF NOT EXISTS `ADBC`.`Produkt` (
  `PRODUKT_ID` INT(11) NULL DEFAULT NULL ,
  `Name` VARCHAR(45) NULL DEFAULT NULL ,
  `Preis` INT(11) NULL DEFAULT NULL )
ENGINE = InnoDB
DEFAULT CHARACTER SET = utf8
COLLATE = utf8_general_ci;

CREATE  TABLE IF NOT EXISTS `ADBC`.`Warenkorb_has_Produkt` (
  `Warenkorb_WARENKORB_ID` INT(11) NULL DEFAULT NULL ,
  `Produkt_PRODUKT_ID` INT(11) NULL DEFAULT NULL ,
  `WARENKORB_HAS_PRODUKT_ID` INT(11) NULL DEFAULT NULL ,
  `Datum` DATE NULL DEFAULT NULL )
ENGINE = InnoDB
DEFAULT CHARACTER SET = utf8
COLLATE = utf8_general_ci

PARTITION BY RANGE COLUMNS(Datum) (
    PARTITION quartal1 VALUES LESS THAN ('2011-04-01'),
    PARTITION quartal2 VALUES LESS THAN ('2011-07-01'),
    PARTITION quartal3 VALUES LESS THAN ('2011-10-01'),
    PARTITION quartal4 VALUES LESS THAN MAXVALUE
);

SET SQL_MODE=@OLD_SQL_MODE;
SET FOREIGN_KEY_CHECKS=@OLD_FOREIGN_KEY_CHECKS;
SET UNIQUE_CHECKS=@OLD_UNIQUE_CHECKS;
\end{lstlisting}

\begin{lstlisting}[caption=Tabellenerzeugung mit Sub-Partitioning, firstnumber=1]{code:createpartsub}
SET @OLD_UNIQUE_CHECKS=@@UNIQUE_CHECKS, UNIQUE_CHECKS=0;
SET @OLD_FOREIGN_KEY_CHECKS=@@FOREIGN_KEY_CHECKS, FOREIGN_KEY_CHECKS=0;
SET @OLD_SQL_MODE=@@SQL_MODE, SQL_MODE='TRADITIONAL';

CREATE SCHEMA IF NOT EXISTS `ADBC` DEFAULT CHARACTER SET utf8 COLLATE utf8_general_ci ;

USE `ADBC`;

CREATE  TABLE IF NOT EXISTS `ADBC`.`Kunde` (
  `KUNDE_ID` INT(11) NULL DEFAULT NULL ,
  `Name` VARCHAR(45) NULL DEFAULT NULL ,
  `Kundennummer` VARCHAR(45) NULL DEFAULT NULL )
ENGINE = InnoDB
DEFAULT CHARACTER SET = utf8
COLLATE = utf8_general_ci;

CREATE  TABLE IF NOT EXISTS `ADBC`.`Warenkorb` (
  `WARENKORB_ID` INT(11) NULL DEFAULT NULL ,
  `Kunde_KUNDE_ID` INT(11) NULL DEFAULT NULL )
ENGINE = InnoDB
DEFAULT CHARACTER SET = utf8
COLLATE = utf8_general_ci;

CREATE  TABLE IF NOT EXISTS `ADBC`.`Produkt` (
  `PRODUKT_ID` INT(11) NULL DEFAULT NULL ,
  `Name` VARCHAR(45) NULL DEFAULT NULL ,
  `Preis` INT(11) NULL DEFAULT NULL )
ENGINE = InnoDB
DEFAULT CHARACTER SET = utf8
COLLATE = utf8_general_ci;

CREATE  TABLE IF NOT EXISTS `ADBC`.`Warenkorb_has_Produkt` (
  `Warenkorb_WARENKORB_ID` INT(11) NULL DEFAULT NULL ,
  `Produkt_PRODUKT_ID` INT(11) NULL DEFAULT NULL ,
  `WARENKORB_HAS_PRODUKT_ID` INT(11) NULL DEFAULT NULL ,
  `Datum` DATE NULL DEFAULT NULL )
ENGINE = InnoDB
DEFAULT CHARACTER SET = utf8
COLLATE = utf8_general_ci

    PARTITION BY RANGE COLUMNS(Datum)
    SUBPARTITION BY HASH( TO_DAYS(Datum) ) (
        PARTITION quartal1 VALUES LESS THAN ('2011-04-01') (
            SUBPARTITION s0,
            SUBPARTITION s1,
            SUBPARTITION s2,
            SUBPARTITION s3
        ),
        PARTITION quartal2 VALUES LESS THAN ('2011-07-01') (
            SUBPARTITION s4,
            SUBPARTITION s5,
            SUBPARTITION s6,
            SUBPARTITION s7
        ),
        PARTITION quartal3 VALUES LESS THAN ('2011-10-01') (
            SUBPARTITION s8,
            SUBPARTITION s9,
            SUBPARTITION s10,
            SUBPARTITION s11
        ),
        PARTITION quartal4 VALUES LESS THAN MAXVALUE (
            SUBPARTITION s12,
            SUBPARTITION s13,
            SUBPARTITION s14,
            SUBPARTITION s15
        )
    );


SET SQL_MODE=@OLD_SQL_MODE;
SET FOREIGN_KEY_CHECKS=@OLD_FOREIGN_KEY_CHECKS;
SET UNIQUE_CHECKS=@OLD_UNIQUE_CHECKS;
\end{lstlisting}

\begin{lstlisting}[caption=Indexerstellung als Hash, firstnumber=1]{code:createindexhash}
CREATE INDEX kunde_idx
    USING HASH
    ON adbc.kunde (KUNDE_ID);
    
CREATE INDEX produkt_idx
    USING HASH
    ON adbc.produkt (PRODUKT_ID);
    
CREATE INDEX warenkorb_idx
    USING HASH
    ON adbc.warenkorb (WARENKORB_ID);
    
CREATE INDEX warenkorb_kunde_idx
    USING HASH
    ON adbc.warenkorb (Kunde_KUNDE_ID);
    
CREATE INDEX warenkorb_has_produkt_idx
    USING HASH
    ON adbc.warenkorb_has_produkt (WARENKORB_HAS_PRODUKT_ID);
    
CREATE INDEX warenkorb_has_produkt_Warenkorb_WARENKORB_ID_idx
    USING HASH
    ON adbc.warenkorb_has_produkt (Warenkorb_WARENKORB_ID);

CREATE INDEX warenkorb_has_produkt_Produkt_PRODUKT_ID_idx
    USING HASH
    ON adbc.warenkorb_has_produkt (Produkt_PRODUKT_ID);
\end{lstlisting}

\begin{lstlisting}[caption=Indexerstellung als B-Tree, firstnumber=1]{code:createpartbtree}
CREATE INDEX kunde_idx
    USING BTREE
    ON adbc.kunde (KUNDE_ID);
    
CREATE INDEX produkt_idx
    USING BTREE
    ON adbc.produkt (PRODUKT_ID);
    
CREATE INDEX warenkorb_idx
    USING BTREE
    ON adbc.warenkorb (WARENKORB_ID);
    
CREATE INDEX warenkorb_kunde_idx
    USING BTREE
    ON adbc.warenkorb (Kunde_KUNDE_ID);
    
CREATE INDEX warenkorb_has_produkt_idx
    USING BTREE
    ON adbc.warenkorb_has_produkt (WARENKORB_HAS_PRODUKT_ID);
    
CREATE INDEX warenkorb_has_produkt_Warenkorb_WARENKORB_ID_idx
    USING BTREE
    ON adbc.warenkorb_has_produkt (Warenkorb_WARENKORB_ID);

CREATE INDEX warenkorb_has_produkt_Produkt_PRODUKT_ID_idx
    USING BTREE
    ON adbc.warenkorb_has_produkt (Produkt_PRODUKT_ID);
\end{lstlisting}


\end{appendix}

\newpage
\chapter{Arbeitsaufteilung}


\begin{table}[h] \begin{flushleft}  \begin{tabular}{|l||c|c|c|c|c|c|}
\hline
\textbf{Arbeit}		&	\textbf{C. Ochmann}	& \textbf{I. K�rner}  \\ \hline \hline
Abstract   	      &                     & 0       \\
Einleitung  &                             		      & ~\ref{Einleitung} \\
Aufgabenstellung&                                  & ~\ref{Aufgabenstellung}  \\
Forschungsgegenstand&                              & ~\ref{RelevanzDesForschungsgegenstandes} \\ 
akt. Wissensstand&                                      & ~\ref{DerAktuelleWissensstand}  \\ 
PostgreSQL&                                         & ~\ref{PostgreSQL} \\
Was ist ein Ausf�hrungsplan? &                      &~\ref{WasIstEinAusfuehrungsplan} \\
Die Abarbeitung von Abfragen in PostgreSQL&  ~\ref{Abarbeitung}       & \\ 
Ausf�hrungsplan ver�ndern& ~\ref{Ausfuehrungsplan}        & \\ 
Kostenparameter&   ~\ref{Kostenparameter}      & \\ 
Indexe&          ~\ref{Indexe}                                    & \\ 
Indextypen&    ~\ref{Indextypen}     & \\ 
Pl�ne mit Bitmap Index Scan&  ~\ref{Plaene}       & \\ 
Plananalyse&   ~\ref{Plananalyse}      & \\ 
Planvergleich&  ~\ref{Planvergleich}       & \\ 
Statistiken&  ~\ref{Statistiken}       & \\ 
Die drei Scan-Algorithmen&                              & ~\ref{DieDreiScanAlgorithmen} \\ 
Die drei Join-Algorithmen&         											& ~\ref{DieDreiJoinAlgorithmen}\\ 
Datenbankentwurf&         															& ~\ref{Datenbankentwurf} \\ 
Der Datengenerator&         																& ~\ref{DerDatengenerator} \\ 
Datenbankabfragen&         															& ~\ref{Datenbankabfragen} \\ 
Was ist ein Hints-System?&        												 & ~\ref{WasIstEinHintsSystem} \\ 
Planerverwirrung&        																		 & ~\ref{Planerverwirrung} \\ 
Reihenfolge von Joins erzwingen&     										    & ~\ref{ReihenfolgeVonJoinsErzwingen} \\ 
Ausf�hrungspl�ne f�r Queries mit und ohne Index&         & ~\ref{AusfuehrungsplaeneFuerQueriesMitUndOhneIndex} \\ 
Was tun bei langsamen Ausf�hrungspl�nen?&       			  & ~\ref{WasTunBeiLangsamenAusfuehrungsplaenen} \\ 
Zusammenfassung&         																& ~\ref{Zusammenfassung} \\ 
Ausblick&        																					 & ~\ref{Ausblick} \\ 
\hline \hline
\end{tabular} \end{flushleft} \caption{Aufteilung von Kapitel 2} \end{table}



\newpage
\chapter{Eigenst�ndigkeitserkl�rung}
Hiermit erkl�re ich, dass ich diese Arbeit selbst�ndig verfasst habe. Mir ist bekannt, dass jede Form des Plagiats mit der Note 5 (Betrugsversuch) bewertet wird.

\begin{tabular}{@{}p{6.0cm}p{6.0cm}}	  		  		 	
	  		 & \\
	  		 & \\
				  			\textbf{Ochmann, Christof} &               Unterschrift:\\				 
				 &\\
				 & \\
				  			\textbf{K�rner, Ingo}   	&                     Unterschrift:\\			
\end{tabular}
\end{document}
