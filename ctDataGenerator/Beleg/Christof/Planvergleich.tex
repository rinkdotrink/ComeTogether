\section{Planvergleich}\label{Planvergleich}

Mit dem Befehl \textbf{EXPLAIN ANALYZE} kann man die gesch�tzten Kosten mit den Ergebnissen der Ausf�hrung vergleichen. Die Anfrage wird auch auch tats�chlich ausgef�hrt:

\begin{verbatim}
"Hash Join  (cost=480.53..506.29 rows=1 width=38) 
		(actual time=19.767..20.204 rows=1 loops=1)"
"  Hash Cond: (u.userid = p.userid)"
"  ->  Seq Scan on "user" u  
		(cost=0.00..22.00 rows=1000 width=38) 
		(actual time=0.029..0.279 rows=1000 loops=1)"
"  ->  Hash  (cost=480.52..480.52 rows=1 width=8) 
				(actual time=19.439..19.439 rows=1 loops=1)"
"        Buckets: 1024  Batches: 1  Memory Usage: 1kB"
"        ->  Hash Join  (cost=279.01..480.52 rows=1 width=8) 
							(actual time=9.220..19.428 rows=1 loops=1)"
"              Hash Cond: (p.eventid = e.eventid)"
"              ->  Seq Scan on participation p  
									(cost=0.00..164.00 rows=10000 width=16) 
									(actual time=0.037..4.178 rows=10000 loops=1)"
"              ->  Hash  (cost=279.00..279.00 rows=1 width=8) 
									(actual time=9.109..9.109 rows=1 loops=1)"
"                    Buckets: 1024  Batches: 1  Memory Usage: 1kB"
"                    ->  Seq Scan on event e  
										(cost=0.00..279.00 rows=1 width=8) 
										(actual time=0.040..9.102 rows=1 loops=1)"
"                          Filter: (eventname = 'event1'::text)"
"Total runtime: 20.386 ms"

\end{verbatim}

Die Zahlen in der ersten Klammer sind dieselben wie bei \textbf{EXPLAIN} und die Daten in der zweiten Klammer enthalten analog zu den Kosten die tats�chliche Ausf�hrungszeit sowie die tats�chliche Anzahl der Zeilen. Der Parameter loops zeigt an wie oft dieser Teilplan ausgef�hrt wird, spielt jedoch erst bei Joins eine Rolle, denn bei einfachen Anfragen ist die Zahl logischerweise eine 1. 

Bei \textbf{EXPLAIN ANALYZE} ist es am wichtigsten darauf zu achten, ob die Zeilensch�tzung richtig war. Bei gr��eren Abweichungen m�ssen die Statistiken verbessert werden.