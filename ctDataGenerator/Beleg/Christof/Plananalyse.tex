\section{Plananalyse}\label{Plananalyse}

Mit dem Befehl \textbf{EXPLAIN} gefolgt von der eigentlichen Anfrage k�nnen die Ausf�hrungpl�ne angesehen werden:

z.B. Anfrage ohne Indexe:
\begin{verbatim}
EXPLAIN Select u.userId, u.name, u.email, u.gender, u.birthday
from public.user u, public.event e, public.participation p
where e.eventname = 'event1'
AND e.eventid = p.eventid
AND p.userid = u.userid;
\end{verbatim}

\begin{verbatim}
"Hash Join  (cost=480.53..506.29 rows=1 width=38)"
"  Hash Cond: (u.userid = p.userid)"
"  ->  Seq Scan on "user" u  (cost=0.00..22.00 rows=1000 width=38)"
"  ->  Hash  (cost=480.52..480.52 rows=1 width=8)"
"        ->  Hash Join  (cost=279.01..480.52 rows=1 width=8)"
"              Hash Cond: (p.eventid = e.eventid)"
"              ->  Seq Scan on participation p  
									(cost=0.00..164.00 rows=10000 width=16)"
"              ->  Hash  (cost=279.00..279.00 rows=1 width=8)"
"                    ->  Seq Scan on event e  
													(cost=0.00..279.00 rows=1 width=8)"
"                          Filter: (eventname = 'event1'::text)"

\end{verbatim}

\begin{itemize}
\item Die erste Zahl nach dem "`cost"' sind die Startkosten. Es ist der gesch�tzte Aufwand, den der Executor investieren muss, bevor der Planknoten Ergebnisse produzieren kann. Eine 0.00 bedeutet, dass bei einem sequenziellen Scan die Ergebnisse sofort ausgegeben werden, sobald der Scan ausgef�hrt wird.  

\item Die zweite Zahl ist der gesch�tzte Aufwand f�r die Abarbeitung des Planknotens und somit die interessante von den beiden Zahlen.

\item Die Zeilenzahl(rows) ist eine Sch�tzung �ber die Anzahl der ausgegebenen Ergebniszeilen. Anhand dieser Zahl wird die Abw�gung getroffen, ob ein Indexscan g�nstiger ist oder auch nicht und ob andere Planvarianten infrage kommen. 

\item Die letzte Zahl gibt die Gr��e der Ergebniszeile in Bytes an(width). Man kann mit Hilfe der Zeilenzahl(rows) den Speicherbedarf der Ergebnismenge vorhersehen(rows * width).
\end{itemize}

Die gesch�tzten Kosten f�r einen sequenziellen Scan der Tabelle "`user"', die in unserem Beispiel auf 22.00 gesch�tzt werden,  lassen sich mit Hilfe folgender Anfrage in PostgreSQL leicht berechnen: \\ 

SELECT relpages, reltuples FROM pg\_class WHERE relname = 'user'
\\

Das Ergebnis dieser Anfrage ist: \\
relpages: 12 \\
reltuples: 1000

Das bedeutet, dass die Tabelle "`user"' aus 12 Diskpages besteht und 1000 Zeilen enth�lt.

Der gesch�tzte Aufwand f�r die Abarbeitung des Planknotens, der nur aus einem sequenziellen Scan besteht (Seq Scan on "`user"' u  (cost=0.00..22.00 rows=1000 width=38)
), wird wie folgt berechnet:

(Anzahl der Diskpages(je 8 KByte) x seq\_page\_cost) + ( rows x cpu\_tuple\_cost)) 
In unserem Beispiel1:
(12 * 1.0) + (1000 * 0.01) = 22 

Also die Anzahl der Seiten mal die Kosten f�r das sequenzielle Lesen einer Page plus die Anzahl der Zeilen mal die Kosten f�r die Abarbeitung einer Zeile in der CPU.