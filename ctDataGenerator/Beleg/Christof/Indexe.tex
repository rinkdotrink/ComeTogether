\section{Indexe}

Ein Index auf eine Spalte lohnt sich nur bei hoher Selektivit�t. Wenn die Selektivit�t nicht hoch ist, muss sowieso die gesamte Tabelle durchgegangen werden, d.h. jeder Block angefasst werden.
Wenn der Planer mit analyze genug Statistiken hat, entscheidet er, ob er den Index verwendet oder nicht.

Ein Index lohnt sich nicht, wenn man viele Ergebnisse erwartet und einen nur die ersten zehn Zeilen interessieren. Denn dann kann sich der Planer das durchhashen und sortieren ersparen.
Der Planer sollte den Index nicht verwenden, wenn ein Limit von z.B. zehn angegeben wurde.

\subsection{Indextypen}

Von PostgreSQL werden derzeit vier Indextypen unterst�tzt:

\begin{itemize}

\item \textbf{B-tree}: \\
Es ist die Implementierung des B+-Baums und der Standardindextyp. Er kann alle Anfragen mit den Vergleichsoperatoren( <, <=, =, >=, >) und den Konstrukten wie BETWEEN und IN bearbeiten.

\item \textbf{Hash}: \\ 
Dieser Indextyp verwendet eine Hashtabelle und bedient nur Anfragen mit dem Gleichheitsoperator(=). Er bietet keine bessere Geschwindigkeit als B-tree und wird heutzutage haupts�chlich nur bei Experimenten verwendet. 

\item \textbf{GiST}: \\
Bei GiST handelt es sich um eine universelle Schnittstelle, um verschiedene anwendungsspezifische Indextypen selbst definieren zu k�nnen. Zwei Anwendungen davon sind PostGIS und OpenFTS(Open Source Full Textj Search). Man verwendet diesen Indextyp um beispielsweise Geodaten zu sortieren oder bei der Volltextsuche.

\item \textbf{GIN}: \\
Hierbei handelt es sich um einen invertierten Index, der Werte mit mehreren Schl�sseln wie beispielsweise Arrays und Listen indizieren kann.

\end{itemize}

\subsection{Pl�ne mit Bitmap Index Scan}

Der reine Indexscan eignet sich am besten dort, wo die Selektivit�t sehr hoch ist und der reine Sequential Scan sollte bei niedriger Selektivit�t verwendet werde, also dann wenn die Tabelle komplett oder fast komplett gelesen werden soll. Der Bitmap Index Scan dagegen soll das gesamte Spektrum zwischen Index- und Sequential Scan abdecken, also die Vorteile von beiden Indextypen kombinieren. 

z.B.
\begin{verbatim}
EXPLAIN Select *
from public.user u
where u.gender = 'w'
AND u.birthday between '01.01.1986' AND '31.12.2013'
\end{verbatim}

Query Plan:
\begin{verbatim}
"Bitmap Heap Scan on "user" u  (cost=64.62..236.66 rows=1412 width=61)"
"  Recheck Cond: ((birthday >= '1986-01-01'::date) 
		AND (birthday <= '2013-12-31'::date))"
"  Filter: (gender = 'w'::text)"
"  ->  Bitmap Index Scan on user_birthday  
				(cost=0.00..64.27 rows=2802 width=0)"
"        Index Cond: ((birthday >= '1986-01-01'::date) 
					AND (birthday <= '2013-12-31'::date))"
\end{verbatim}
Wie man bei diesem Ausf�hrungsplan sehen kann, besteht der Bitmap Index Scan aus zwei Schritten. Zuerst werden alle Treffer aus dem Index ermittelt und in einer Bitmap im RAM zwischengespeichert und sortiert. Anschlie�end werden die Treffer beim Bitmap Heap Scan in der Tabelle der Reihe nach aufgesucht. Die Spr�nge zwischen Tabelle und Index werden somit verhindert. Der Bitmap Index Scan liefert dem Bitmap Heap Scan die Eingabe und ist ihm somit untergeordnet, was man bei einem Ausf�hrungsplan an dem Pfeil erkennen kann. Die Startkosten f�r den Bitmap Heap Scan sind gr��er null, d.h., dass der Bitmap Index Scan abgeschlossen sein muss, bevor der Bitmap Heap Scan seine Arbeit beginnen kann.
