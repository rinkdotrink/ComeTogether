\section{Die Abarbeitung von Abfragen in PostgreSQL}
\begin{enumerate} 
\item \textbf{Empfang des SQL-Befehls} \\
Nachdem der SQL-Befehl �ber eine Netzwerkverbindung �bertragen wurde, findet die Kodierungsumwandlung statt, 
und die weiteren Phasen der Abarbeitung sehen den Befehl in der Serverkodierung. Hierbei gibt es nur sehr
geringe Optimierungsm�glichkeiten. Es k�nnen theoretisch CPU-Zyklen gespart werden, wenn die Clientkodierung 
gleich der Serverkodierung ist, ansonsten wird eine Konvertierung durchgef�hrt. Diese Auswirkungen sind jedoch sehr gering.
Der Parameter \textit{client\_encoding} informiert den Server dar�ber, welche Kodierung die ankommenden Befehle haben 
und welche Kodierung das Anfrageergebnis haben soll, welches an den Client gesendet wird. Die Voreinstellung 
gibt an welche Kodierung der Server intern verwendet.

\item \textbf{Parser} \\
In dieser Abarbeitungsphase wird die kodierte Befehlszeichenkette durch einen internen Parse-Baum dargestellt. 
Des Weiteren wird die Befehlszeichenkette auf semantische Bedingungen �berpr�ft und etwas bearbeitet. Die SQL-Befehle werden dann 
aufgeteilt in sogenannte optimierbare Anweisungen(SELECT, INSERT, UPDATE und DELETE) und Hilfsanweisungen.
Die Hilfsanweisungen werden sp�ter direkt ausgef�hrt und sie erzeugen keine Ausgabe. Dagegen kommen die optimierbaren 
Anweisungen in den Rewriter. F�r den Parser gibt es von der Anwenderseite keine M�glichkeit die Geschwindigkeit 
zu optimieren. 


\item \textbf{Query Rewriter} \\
Der Rewriter wendet die Anfrageumschreibregeln(Query Rewrite Rules) an. Dabei werden die Sichten(Views) und andere
benutzerdefinierte Regeln aufgel�st, in die Anfrage eingebaut und im Parse-Baum ersetzt. Da der Rewriter 
vor dem Planer angesiedelt ist, bekommt der Planer es nicht mit, ob die Anfrage aus einer Sicht kam oder nicht. Mit der 
Erstellung einer Sicht hat man somit keinen Optimierungsvorteil.

\item \textbf{Planer / Optimizer} \\
Der Planer bekommt den m�glicherweise umgeschriebenen Parse-Baum und hat die Aufgabe einen Ausf�hrungsplan(execution plan) zu erstellen, der ebenfalls ein Baum ist.
Der Ausf�hrungsplan beschreibt wie auf die Tabellen zugegriffen werden soll, also welche Indexe und Join-Algorithmen verwendet werden sollen und in welcher Reihenfolge. Es soll m�glichst der optimalste und schnellste Ausf�hrungsplan gefunden werden. 

\item \textbf{Executor} \\ 
Der vom Planer auserw�hlte Ausf�hrungsplan wird vom Executor ausgef�hrt. Dabei werden Zugriffsrechte auf Tabellen und andere Objekte sowie Constraints gepr�ft. Die Laufzeit der Ausf�hrung h�ngt nicht nur davon ab ob der Plan gut ist, sondern auch von der gesamten Systemkonfiguration.

\end{enumerate}