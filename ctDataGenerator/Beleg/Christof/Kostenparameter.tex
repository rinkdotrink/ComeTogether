\section{Kostenparameter}

Au�er den Plantypen gibt es noch die Kostenparameter, die den erwarteten I/O und CPU-Aufwand z�hlen. �ndert man diese Konfigurationsparameter, dann werden die Anfragen andere Plankosten haben und es wird eventuell ein anderer, g�nstigerer Ausf�hrungsplan gew�hlt. Man muss die Kostenfaktoren so einstellen, dass der Plan dem Planer auch am schnellsten erscheint und somit ausgew�hlt wird.

Die Voreinstellungen dieser Parameter stellen das ungef�hre Verhalten eines typischen Computers dar und m�ssen lediglich an das eigene System angepasst werden. Jedoch gibt es kein Benchmark-Programm, mit dem man diese Einstellungen automatisch vornehmen k�nnte. 
In der Regel sind keine weitreichenden �nderungen an diesen Parametern notwendig, es sein denn man nutzt diese als Mittel um einen anderen Ausf�hrungsplan zu erzwingen. Die �nderungen sollten in kleinen Schritten und mit ausf�hrlichen Tests f�r alle in Frage kommenden Anfragen vorgenommen werden, denn eine falsche Konfiguration kann bei bestimmten Anfragen zu einem hohen Ressourcenverbrauch f�hren. Die wichtigsten Kostenparameter sind:

\begin{itemize}

\item seq\_page\_cost: \\
Wird eine Page von der Platte gelesen, entsteht Aufwand, der mit diesem Parameter definiert ist. Er gibt die Kosten f�r das sequenziellen Lesen einer Page(typischerweise 8kb) w�hrend einer laufenden Leseanweisung an, die Voreinstellung f�r diesen Parameter ist 1.0.

\item random\_page\_cost: \\
Der Parameter gibt an, um wie viel aufw�ndiger es ist, Pages in zuf�lliger Reihenfolge zu lesen, wie es bei Indexscans vorkommt (gemessen am sequentiellem Lesen von Pages). Die Voreinstellung  ist 4.
Die Parameter seq\_page\_cost und random\_page\_cost kann man sich vereinfacht als Kosten f�r sequenzielle Scans und Indexscans vorstellen. Indexscans sind langsamer als das sequenzielle Lesen, daf�r aber muss man in der Regel bei den Indexscans nicht so viele Seiten lesen. Mit diesen Parametern entscheidet der Planer zwischen Index und nicht Index.
Die Voreinstellung besagt, dass ein Indexscan viermal langsamer ist als der sequenzielle Scan. Dieser Wert kann jedoch bei schnellen Festplatten zu hoch sein und es wird empfohlen den Wert pauschal auf 2.0 oder 3.0 herunterzusetzen.
Wenn sich die Datenbank komplett im Hauptspeicher befindet, ist es sinnvoll die beiden Parameter gleichzusetzen, da der Aufwand in diesem Fall gleich ist.

\item cpu\_tuple\_cost: \\
Hiermit werden die Kosten f�r das Verarbeiten einer Tabellenzeile durch die CPU definiert. Die Voreinstellung ist 0.01 und dieser Wert wird in der Regel sehr selten ver�ndert.

\item cpu\_index\_tuple\_cost: \\ 
Dieser Parameter definiert die Kosten f�r das Verarbeiten eines Indexeintrags durch die CPU w�hrend eines Indexscans. Auch dieser Parameter wird sehr selten ver�ndert und ist auf 0.005 voreingestellt.

\item cpu\_operator\_cost: \\
Dieser Konfigurationsparameter gibt die Kosten f�r das Ausf�hren eines Operators oder einer Funktion an. Die Voreinstellung ist 0.0025 und wird in der Regel sehr selten ver�ndert.

\item effective\_cache\_size: \\ 
Dieser Parameter gibt an, wie viel Cache-Speicher f�r eine Anfrage vom Betriebssystem bereitgestellt werden. Es ist nur eine Vermutung und es wird kein Speicher angelegt. Der Planer sieht anhand dieses Parameters ob ein Index im Hauptspeicher h�chstwahrscheinlich gecached ist, oder noch von der Festplatte gelesen werden muss. Daraufhin entscheidet der Planer ob ein indexbasierter Zugriffspfad zu w�hlen ist oder ob nicht eine andere Alternative kosteng�nstiger ist. Es hoher Wert sorgt also daf�r, dass mehr Indexscans verwendet werden. Die Voreinstellung betr�gt 128 MByte.

\end{itemize}