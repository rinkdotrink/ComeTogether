\section{Statistiken}

Der Planer sammelt Statistiken um die Anzahl der zu erwartenden Ergebniszeilen zu ermitteln. Relevant dabei ist die Anzahl der Zeilen in der Tabelle, sowie die Anzahl der von der Tabelle belegten Bl�cke. Diese Daten stehen in der Systemtabelle pg\_class und der Planer kann mit Hilfe dieser Informationen die Anzahl der Zeilen f�r einen sequenziellen Scan �ber eine Tabelle, sowie den Speicher- und I/O-Aufwand ermitteln. Aktuell gehalten werden die Statistiken von den Befehlen \textbf{VACUUM} und \textbf{ANALYZE}, sowie von \textbf{CREATE INDEX}.
Meistens jedoch werden die Tabellen nicht komplett gelesen, sondern m�ssen nach bestimmten Kriterien z.B. mit der WHERE-Klausel ausgew�hlt werden. Dazu muss der Planer weitere Statistiken sammeln, die in der Tabelle pg\_statistics stehen und mit der folgenden Anfrage eingesehen werden k�nnen: \\

SELECT * FROM pg\_stats WHERE tablename = 'user' AND attname = 'birthday'
\\

Das Ergebnis dieser Anfragen sind N�herungswerte und beziehen sich immer nur auf eine Spalte. Die interessantesten sind: 

\begin{itemize}

\item  n\_distinct = 80: Die Zahl bedeutet, dass die Tabelle 80 unterschiedliche Werte enth�lt. 
\item  most\_common\_vals = 1951-02-01, 1947-02-01, usw.: Ein Array mit den h�ufigsten Werten in der Tabelle.
\item correlation = 0.016471: Eine Korrelation zwischen der logischen Reihenfolge der Werte und der Reihenfolge auf der Festplatte. Ein Wert der nah bei -1 oder +1 liegt bedeutet, dass Indexscans g�nstiger sind, weil die Festplattenzugriffe nicht so weit auseinander liegen.

\end{itemize}