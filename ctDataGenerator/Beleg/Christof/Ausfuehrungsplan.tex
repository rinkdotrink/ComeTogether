\section{Ausf�hrungsplan ver�ndern}\label{Ausfuehrungsplan}
In PostgreSQL hat der Anwender keine M�glichkeit selbst einen Plan vorzugeben, aber
mit bestimmten Parametern kann man den Anfrageplaner beeinflussen und somit den Ausf�hrungsplan ver�ndern. Man kann einzelne Plantypen ausschalten und den Planer dazu bringen nur die aktivierten Plantypen zu verwenden. Die Parameter werden w�hrend einer Datenbanksitzung mit dem SET-Befehl gesetzt: 
\\
z.B. \textbf{SET enable\_seqscan TO off} \\


In der Voreinstellung sind alle Plantypen an und k�nnen vom Planer benutzt werden. Die folgende Auflistung zeigt die verschiedenen Plantypen und ihre Bedeutung: \\
\begin{itemize}
\item \textbf{enable\_seqscan}: \\
Wenn der Planer zu sequenziellen Pl�nen tendiert, kann man mit diesem Parameter die Verwendung von indexbasierten Pl�nen erzwingen. Die sequenziellen Pl�ne werden zwar nicht ganz abgeschaltet, aber sie werden extrem hoch bewertet und erscheinen zu aufwendig f�r den Planer um sie zu nutzen. Auf die selbe Art und Weise k�nnen die folgenden Parameter den Planer beeinflussen: 

\item \textbf{enable\_indexscan}: \\
Hiermit werden die indexbasierten Pl�ne aktiviert bzw. deaktiviert.

\item \textbf{enable\_bitmapscan}: \\
Auch das Aktivieren und Deaktivieren von Bitmap Index Scans ist m�glich.

\item \textbf{enable\_nestloop}:\\
Aktivierung bzw. Deaktivierung von Nested-Loop-Joins. 

\item \textbf{enable\_hashjoin}: \\
Aktivierung bzw. Deaktivierung von Hash-Join-Plantypen.

\item \textbf{enable\_mergejoin}: \\
Aktivierung bzw. Deaktivierung von Merge-Join-Plantypen.

\item \textbf{enable\_hashagg}: \\
Aktivierung bzw. Deaktivierung von Hash-basierter Aggregierung.

\item \textbf{enable\_sort}: \\
Sortieroperationen werden vom Planer nicht ber�cksichtigt wenn dieser Parameter deaktiviert ist. Es kann aber vorkommen, dass sie trotzdem im Anfrageplan vorkommen wenn keine Alternativen herangezogen werden k�nnen.

\item \textbf{enable\_tidscan}: \\
Aktivierung bzw. Deaktivierung von TID-basierten Plantypen.
\end{itemize}