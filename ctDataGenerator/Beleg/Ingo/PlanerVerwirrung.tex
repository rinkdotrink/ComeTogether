\section{Planerverwirrung}\label{Planerverwirrung}
Wie k�nnen Ausf�hrungspl�ne noch beschleunigt werden? In PostgreSQL durch Umschreiben nur sehr begrenzt. Es ist zwar m�glich, die Queries auf verschiedene Weisen zu schreiben, aber die Schreibweise hat auf den Optimizer nur in seltenen F�llen Einfluss.
Wie sehen diese F�lle aus?
Es gibt Querykonstruktionen\footnote{http://www.postgresql.org/docs/current/interactive/explicit-joins.html}, die den Planner verwirren k�nnen und somit zu langsamen Ausf�hrungspl�nen f�hren w�rden.

Wird der Operator $>$ oder der Operator $<$\ verwendet, kann der Planer die Tabellen nicht mehr mit Hilfe eines Hash-JOIN verkn�pfen:

\begin{lstlisting}[caption=kein Hash Join m�glich, firstnumber=1, label=keinHashJoin]{code:keinHashJoin}
Select u.userId, p.eventid
from public.user u, public.event e, public.participation p
where e.eventid < p.eventid + 1
AND e.eventid > p.eventid - 1
AND p.userid < u.userid + 1
AND p.userid > u.userid - 1;
\end{lstlisting}

\begin{lstlisting}[caption=Ausf�hrungsplan nested loop join, firstnumber=1, label=aSTRAIGHTJOIN]{code:aSTRAIGHTJOIN}
"Nested Loop  (cost=0.00..283418724402242.31 rows=1192507407395482 width=16)"
"  Join Filter: ((e.eventid < (p.eventid + 1)) AND (e.eventid > (p.eventid - 1)))"
"  ->  Nested Loop  (cost=0.00..2641025223.00 rows=11111111111 width=16)"
"        Join Filter: ((p.userid < (u.userid + 1)) AND (p.userid > (u.userid - 1)))"
"        ->  Seq Scan on participation p  (cost=0.00..22739.00 rows=1000000 width=16)"
"        ->  Materialize  (cost=0.00..3125.00 rows=100000 width=8)"
"              ->  Seq Scan on "user" u  (cost=0.00..2234.00 rows=100000 width=8)"
"  ->  Materialize  (cost=0.00..52708.96 rows=965931 width=8)"
"        ->  Seq Scan on event e  (cost=0.00..44105.31 rows=965931 width=8)"
\end{lstlisting}

besser:

\begin{lstlisting}[caption=Hash Join m�glich, firstnumber=1, label=HashJoinMoeglich]{code:HashJoinMoeglich}
explain Select u.userId, p.eventid
from public.user u, public.event e, public.participation p
where e.eventid = p.eventid
AND p.userid = u.userid;
\end{lstlisting}

\begin{lstlisting}[caption=Ausf�hrungsplan hash join, firstnumber=1, label=ahash]{code:ahash}
"Hash Join  (cost=43997.00..146580.72 rows=965931 width=16)"
"  Hash Cond: (p.userid = u.userid)"
"  ->  Hash Join  (cost=40122.00..113562.10 rows=965931 width=16)"
"        Hash Cond: (e.eventid = p.eventid)"
"        ->  Seq Scan on event e  (cost=0.00..44105.31 rows=965931 width=8)"
"        ->  Hash  (cost=22739.00..22739.00 rows=1000000 width=16)"
"              ->  Seq Scan on participation p  (cost=0.00..22739.00 rows=1000000 width=16)"
"  ->  Hash  (cost=2234.00..2234.00 rows=100000 width=8)"
"        ->  Seq Scan on "user" u  (cost=0.00..2234.00 rows=100000 width=8)"
\end{lstlisting}

Bei Tabellenspalten, die auf Gleichheit gepr�ft werden, kann der Planer sich f�r einen Hash-Join entscheiden. Wenn beide Operanden in dem Vergleich Ergebnis einer Berechnung sind, ist der Planer gezwungen, auf einen Merge-Join zur�ckzugreifen.

\begin{lstlisting}[caption=merge join, firstnumber=1, label=merge]{code:merge}
explain Select u.userId, p.eventid
from public.user u, public.event e, public.participation p
where (e.eventid + 1) - 1  = (p.eventid + 1) - 1
AND (p.userid + 1) - 1 = (u.userid + 1) -1
\end{lstlisting}

\begin{lstlisting}[caption=Ausf�hrungsplan merge join, firstnumber=1, label=amerge]{code:amerge}
"Merge Join  (cost=136351958.69..60508303947.65 rows=2414827500000 width=16)"
"  Merge Cond: ((((e.eventid + 1) - 1)) = (((p.eventid + 1) - 1)))"
"  ->  Sort  (cost=166544.39..168959.22 rows=965931 width=8)"
"        Sort Key: (((e.eventid + 1) - 1))"
"        ->  Seq Scan on event e  (cost=0.00..44105.31 rows=965931 width=8)"
"  ->  Materialize  (cost=136185414.30..138685414.30 rows=500000000 width=16)"
"        ->  Sort  (cost=136185414.30..137435414.30 rows=500000000 width=16)"
"              Sort Key: (((p.eventid + 1) - 1))"
"              ->  Merge Join  (cost=168485.16..12672485.16 rows=500000000 width=16)"
"                    Merge Cond: ((((u.userid + 1) - 1)) = (((p.userid + 1) - 1)))"
"                    ->  Sort  (cost=11907.32..12157.32 rows=100000 width=8)"
"                          Sort Key: (((u.userid + 1) - 1))"
"                          ->  Seq Scan on "user" u  (cost=0.00..2234.00 rows=100000 width=8)"
"                    ->  Materialize  (cost=156577.84..161577.84 rows=1000000 width=16)"
"                          ->  Sort  (cost=156577.84..159077.84 rows=1000000 width=16)"
"                                Sort Key: (((p.userid + 1) - 1))"
"                                ->  Seq Scan on participation p  (cost=0.00..22739.00 rows=1000000 width=16)"
\end{lstlisting}

Wenn nur ein Operand in dem Vergleich Ergebnis einer Berechnung ist, greift der Planner auf einen nested loop Join zur�ck:

\begin{lstlisting}[caption=nested loop join, firstnumber=1, label=loop]{code:loop}
explain Select u.userId, p.eventid
from public.user u, public.event e, public.participation p
where (e.eventid + 1) - 1  = 5
AND (p.userid + 1) - 1 = 7
\end{lstlisting}

\begin{lstlisting}[caption=Ausf�hrungsplan nested loop join, firstnumber=1, label=aloop]{code:aloop}
"Nested Loop  (cost=0.00..30193621347.37 rows=2415000000000 width=16)"
"  ->  Nested Loop  (cost=0.00..6091095.87 rows=483000000 width=8)"
"        ->  Seq Scan on "user" u  (cost=0.00..2234.00 rows=100000 width=8)"
"        ->  Materialize  (cost=0.00..51373.94 rows=4830 width=0)"
"              ->  Seq Scan on event e  (cost=0.00..51349.79 rows=4830 width=0)"
"                    Filter: (((eventid + 1) - 1) = 5)"
"  ->  Materialize  (cost=0.00..30264.00 rows=5000 width=8)"
"        ->  Seq Scan on participation p  (cost=0.00..30239.00 rows=5000 width=8)"
"              Filter: (((userid + 1) - 1) = 7)"
\end{lstlisting}

Es kann gesagt werden, dass Operanden in einem Vergleich nicht das Ergebnis einer Berechnung sein sollten.
