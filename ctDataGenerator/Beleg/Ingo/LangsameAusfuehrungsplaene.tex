\section{Was tun bei langsamen Ausf�hrungspl�nen?}\label{WasTunBeiLangsamenAusfuehrungsplaenen}

Dem Planer keine falschen Hinweise geben. Keine Hints verwenden.

Autoanalyze aktivieren.
Der Planner kann nur dann optimierte Ausf�hrungspl�ne erzeugen, wenn er gen�gend Statistiken �ber die gef�llten Tabellen besitzt. Deswegen m�ssen regelm��ig Statistiken erstellt werden. In PostgreSQL geht das mit analyze. Analyze sammelt Informationen �ber den F�llstand der Tabellen, die h�ufigsten Werte in jeder Spalten und die wahrscheinliche Verteilung der Werte in einer Spalte.
Mit diesen Statistiken kann dann Planner dann den passenden JOIN-Algorithmus und die passende JOIN-Order w�hlen.
Um nach bestimmten Abst�nden automatisch analyze aufzurufen gibt es autoanalyze.

Autovacuum aktivieren.
Mit autovacuum = on in der postgresql.conf kann autovacuum aktiviert werden.

Indexe verwenden.
Es kann z.B. der Entwickler geeignete Indexe anlegen, sodass das DBMS diese verwenden kann, um performantere Ausf�hrungspl�ne zu erzeugen.

Mehrspaltige Indexe verwenden, wenn dadurch ein performanterer Ausf�hrungs\-plan erzeugt werden kann.

Keine zu intensiven Rechnungen in SQL formulieren.

Partitioning verwenden.

Die Struktur der Tabellen �berdenken, wenn mehr als 8 Tabellen miteinander verkn�pft werden.

Herausfinden, warum der Planner einen langsamen Plan erzeugt, anstatt durch Planner-Hints der Frage aus dem Weg zu gehen.

Es kann sein, dass jemand ung�nstige Werte f�r die Parameter in der postgresql.conf gesetzt hat.

Bei einem zu kleinen Wert f�r $join\_collapse\_limit$ in der postgresql.conf, verbunden mit der expliziten Verkn�pfung von Tabellen mit dem Wort Join, ist der Planner gezwungen, eine vorgegebene aber ung�nstige Join-Reihenfolge zu verwenden.

Um dem Planner nicht ausversehen zu etwas zu zwingen, sollten bei deinem Innerjoin die Tabellen nicht explizit mit dem Wort JOIN verkn�pfen werden, sondern es sollten die Tabellen einfach getrennt durch ein Komma angeben werden:

\begin{lstlisting}[caption=Inner-Join, firstnumber=1, label=gtlt]{code:gtlt}
SELECT * FROM a, b, c WHERE a.id = b.id AND b.ref = c.id;
\end{lstlisting}

Bei einem zu kleinen Wert f�r $from\_collapse\_limit$ in der postgresql.conf, verbunden mit vielen Subqueries, kann der Planner die Subqueries nicht aufl�sen. Dabei sollte der Planner Subqueries zu Joins aufl�sen, da sonst erst das komplette Ergebnis des Subquerys erstellt werden muss, bevor mit der Tabelle weitergearbeitet werden kann.
