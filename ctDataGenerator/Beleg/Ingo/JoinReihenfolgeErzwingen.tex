\section{Reihenfolge von Joins erzwingen}\label{ReihenfolgeVonJoinsErzwingen}
Um den Planner zu der angegebenen Join-Reihenfolge zu zwingen, kann das

$join\_collapse\_limit$\footnote{http://www.postgresql.org/docs/current/interactive/explicit-joins.html} auf 1 gesetzt werden.

Um den Planner zu zwingen, Subqueries nicht in einen JOIN umzuandeln, kann das $from\_collapse\_limit$ auf 1 gesetzt werden.

Der Standardwert f�r $join\_collapse\_limit$ und $from\_collapse\_limit$ ist acht. Bei z.B. zw�lf zu verkn�pfenden Tabellen wird keine vollst�ndige Suche mehr nach der besten Joinreihenfolge ausgef�hrt, sondern eine wahrscheinlichkeitstheoretische genetische Suche die nur noch eine begrenzte Zahl von m�glichen Joinreihenfolgen betrachtet. Die genetische Suche braucht weniger Zeit als die vollst�ndige Suche, findet aber nicht zwangsl�ufig die bestm�gliche Joinreihenfolge.
Ab welchem Schwellwert die genetische Suche aktiv wird, kann bei $geqo\_threshold$ gesetzt werden. Der Standardwert ist zw�lf.

Auch wenn $join\_collapse\_limit$ auf den Wert eins gesetzt wird, wird bei folgendem Select die JOIN-Reihenfolge vom Planer bestimmt:

\begin{lstlisting}[caption=Reihenfolge vom Planer bestimmt, firstnumber=1, label=PlanerBestimmt]{code:PlanerBestimmt}
SELECT * FROM a, b, c WHERE a.id = b.id AND b.ref = c.id;
\end{lstlisting}

Erst wenn zwei Tabellen ausdr�cklich mit dem Wort JOIN verkn�pft werden, zwingt das den Planner, diese zwei Tabellen in der gegebenen Reihenfolge zu verkn�pfen:

\begin{lstlisting}[caption=Reihenfolge vom Entwickler bestimmt, firstnumber=1, label=EntwicklerBestimmt]{code:EntwicklerBestimmt}
SELECT * FROM a CROSS JOIN b CROSS JOIN c WHERE a.id = b.id AND b.ref = c.id;
SELECT * FROM a JOIN (b JOIN c ON (b.ref = c.id)) ON (a.id = b.id);
\end{lstlisting}