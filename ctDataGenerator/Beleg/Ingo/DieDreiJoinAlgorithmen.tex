\section{Die drei Join-Algorithmen}\label{DieDreiJoinAlgorithmen}
In den Knoten des Ausf�hrungsplanes stehen Datenbankoperatoren wie Projek\-tion, Selektion, Kreuzprodukt, Vereinigung, Differenz oder Umbenennung.

Die Hintereinanderausf�hrung der Operationen kartesisches Produkt und Selektion wird Join genannt. Joins werden im DBMS intern �ber Join-Algorithmen realisiert. Join-Algorithmen verkn�pfen Tabellen paarweise miteinander.

In PostgreSQL gibt es drei grundlegende JOIN-Algorithmen:

\begin{enumerate}
\item nested loop join \\
F�r jede Zeile aus der treibenden Tabelle wird die innere Tabelle einmal durchlaufen. Wenn die innere Tabelle indiziert ist, kann sie mit einem Index-Scan durchlaufen werden.
Ein Nested Loop Join kann sehr kostenintensiv werden, wenn er die innere Tabelle mehrmals lesen muss.
Wenn auf der inneren Tabelle ein Index-Scan erfolgen kann, oder die innere Tabelle sehr klein ist, kann sich diese durch vorherige Anfragen im Cache befinden und so schnell abgearbeitet werden.
Wenn es sich um einen sequentiellen Scan handelt, der alle Zeilen vergleicht, dann muss die innere Tabelle u.U. so oft von der Festplatte gelesen werden, wie die treibende Tabelle Zeilen hat.
Wird nur die erste �bereinstimmende Zeile gesucht, ist der Nested Loop Join schneller als andere Joins, die vorher erst ihr komplettes Ergebnis berechnen m�ssen, bevor sie den ersten Treffer zur�ckgeben k�nnen.
Der Nested Loop Join verlangt vor dem Query keinerlei Investition, wie Hashing oder Sortierung.

\item hash join \\
F�r einen Hashjoin m�ssen beide Tabellen als Hash-Tabelle vorliegen. Das setzt vorraus, dass bevor ein Hash Join eingesetzt werden kann, die Tabellen durchgehasht wurden, d.h. ein Hashwert f�r das sp�tere Join-Attribut gebildet wird. Wie der Nested Loop Join ist der Hash-Join besonders performant, wenn der Gr��enunterschied zwischen treibender Tabelle und innerer Tabelle gro� ist und die kleinere Tabelle komplett in den Speicher passt. Dazu wird bei der Ausf�hrung des Hash-Joins die kleinere Hashtabelle in den Arbeitsspeicher geladen, mit dem JOIN-Attribut als Schl�ssel. Dann wird die gr��ere Tabelle gescannt und jeder gefundene Wert wird als Schl�ssel f�r die kleinere Hashtable benutzt.
Ein Hash-Join kann nur dann verwendet werden, wenn die Spalten mit dem = Operator verglichen werden.
Der Performancezugewinn des Hash-Join wird durch einen h�heren Sekund�rspeicherbedarf erkauft, denn die Hashtabellen werden im Tempspace materialisiert. Bei einem Hash Join muss immer ein Materialize erfolgen, der die Buckets erzeugt.
Der Hash-Join eignet sich auch dann, wenn alle L�sungen gebraucht werden und nicht nur z.B. die ersten zehn.

\item merge join \\
Bevor ein Merge Join ausgef�hrt werden kann, m�ssen beide Tabellen nach dem Join-Attributen sortiert werden. Liegen beide Tabellen sortiert vor, werden bei einem Merge Join beide Tabellen parallel gescannt und passende Zeilen werden zusammengef�gt. Sowohl die treibende als auch die innere Tabelle muss nur einmal gescannt werden. Die vorrangegangene Sortierung erfolgt in einem extra Sortierschritt oder durch die Verwendung eines Index, falls das Feld indiziert ist und der JOIN �ber dieses Attribut erfolgt.
Die Sortierung im ersten Fall ist teurer als wenn ein entsprechender Index vorhanden ist.
Werden mehrere Merge-JOINS �ber dasselbe Attribut hintereinander ausgef�hrt, muss nur einmal sortiert werden.
Wie bei dem Hash-Join wird der Performancezugewinn des Merge Joins mit einem h�heren Sekund�rspeicherbedarf erkauft, denn beide Tabellen m�ssen sortiert im Tempspace vorliegen. Der Merge Join ist wie der Hash-Join vor allem dann interessant, wenn alle L�sungen gefunden werden sollen.
Anders als bei Nested Loop Join und Hash-Join spielt bei einem Merge Join der Gr��enunterschied der beiden zu joinenden Tabellen f�r die Ausf�hrungsgeschwindigkeit keine Rolle.
\end{enumerate}