\section{Relevanz des Forschungsgegenstandes}\label{RelevanzDesForschungsgegenstandes}
Der Forschungsgegenstand dieser Arbeit ist, Annahmen �ber erzeugte Ausf�hr\-ungs\-pl�ne von Abfragen zu treffen und gegebenfalls wiederlegte Annahmen zu erkl�ren.
Der Forschungsgegenstand ist relevant, da bisher keine konkreten Aus\-f�hr\-ungs\-pl�ne f�r die gew�hlten Abfragen vorliegen. Ziel dieser Foschung ist es, Ausf�hrungspl�ne zu finden, die die h�chste Ausf�hrungsgeschwindigkeit f�r alle Abfragen bringt. 

Um die optimalen Ausf�hrungspl�ne zu finden, muss sich vertiefend in eine PostgreSQL
eingearbeitet werden. Das geschieht z.B. unter Zuhilfenahme von B�chern und
Online-Ressourcen. In diesen Medien ist der Forschungsstand zur Erstellung von
Ausf�hrungspl�nen dokumentiert. Bei der Anpassung des Datengenerators m�ssen
dar�berhinaus technische Probleme gel�st werden.