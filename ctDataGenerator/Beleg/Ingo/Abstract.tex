\section*{Abstract}\label{Abstract}
Diese Arbeit baut auf das Projekt Datenbankkonfigurationen\footnote{https://dl.dropbox.com/u/608146/ADBC1\%20OLAP.pdf}, dass w�hrend der Vorlesungsreihe ADBC1 erstellt wurde, auf.
Im Projekt Datenbankkonfigurationen wird untersucht, wie sich die Ausf�hrungsgeschwindigkeit von Abfragen steigern l�sst.
Im Projekt Ausf�hrungsplanoptimierung in PostgreSQL wird dar�berhinaus untersucht, welche Ausf�hrungspl�ne, f�r bestimmte Konfigurationen und bestimmte Queries erzeugt werden. Und wie sich diese Pl�ne auf die Ausf�hrungs\-geschwin\-digkeit von SQL-Queries auswirken.
In beiden Projekten wird nur der Bereich OLAP betrachtet.
Diese Arbeit behandelt nur Datenbankkonfigurationen, die Einfluss auf den Ausf�hrungsplan haben. F�r alle Konfigurationen die keinen Einfluss haben, wird der Standardwert von PostgreSQL beibehalten.
Ziel der Arbeit ist, Annahmen �ber die Ausf�hrungsgeschwindigkeiten verschiedener Ausf�hrungspl�ne zu treffen, diese theoretisch zu begr�nden und dann durch Messergebnisse praktisch zu belegen. Anhand der Messergebnisse werden die aufgestellten Hypothesen best�tigt oder wiederlegt.
F�r wiederlegte Hypothesen wird eine Begr�ndung gesucht.